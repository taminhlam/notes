\mainlanguage[en-us]
% Enable hyperlinks
\setupinteraction
  [state=start,
  title={Powerful Ideas Need Love Too !},
  author={Alan Kay},
  style=,
  color=,
  contrastcolor=]

\setuppagenumbering[location={footer,middle}]
%~ \setupbackend[export=yes]
\setupstructure[state=start,method=auto]

\setupwhitespace[medium]



\setuphead[chapter, section, subsection, subsubsection, subsubsubsection, subsubsubsubsection][number=no]



% \setupexternalfigures[directory={images, /home/user/images}]
%~ \setupexternalfigures[directory={[[foobar]]}]
\setupexternalfigures[directory={kay-powerful-ideas}]


\environment style
\environment colorbrouwer


\starttext
\title[Powerful Ideas Need Love Too !]{Powerful Ideas Need Love Too !}

\placeaside[right,high]{}{
  \startaside[width=.3\textwidth]
    \unhyphenated{
      \startitemize[none, fit][stopper={}]
        \item  Alan Kay
        \item October 12, 1995
      \stopitemize }
  \stopaside }
\startabstract
Powerful Ideas Need Love Too !
\stopabstract
\blank[big]

\section[title={Powerful Ideas Need Love
Too !},reference={powerful-ideas-need-love-too}]

\startitemize[packed]
\item
  Alan Kay
\item
  Fellow, Apple Computer Corp.
\item
  {\em Written remarks to a Joint Hearing of the Science Committee and
  the Economic and Educational and Opportunites Committee}
\stopitemize

{\bf October 12, 1995}

Let me start the conversation by showing a video made by the National
Science Foundation at a recent Harvard commencement, in which they asked
some of the graduating seniors and their professors a few simple
questions about what causes the seasons and the phases of the moon. All
were confident about their answers, but roughly 95\letterpercent{} gave
explanations that were not even close to what science has discovered.
Their main theories were that the seasons are caused by the Earth being
closer to the sun in summer, and that the phases of the moon are caused
by the Earth's shadow. Some of the graduates had taken quite a bit of
science in high school and at Harvard. NSF used this to open a
discussion about why science isn't learned well even after years of
schooling. And not learned well even by most of the successful students,
with high SATs, at the best universities, with complete access to
computers, networks, and information.

My reaction was a little different. I kept waiting for the
\quotation{other questions} that NSF should have asked, but they never
did. I got my chance a few weeks later after giving a talk at UCLA. I
asked some of the seniors, first year graduate students, and a few
professors the same questions about the seasons and the phases of the
moon and got very similar results : about 95\letterpercent{} gave bogus
explanations along the same lines as the Harvard students and
professors. But now I got to ask the next questions.

To those that didn't understand the seasons, I asked if they knew what
season it was in South America and Australia when it is summer in North
America. They all knew it was winter. To those that didn't understand
the phases of the moon, I asked if they had ever seen the moon and the
sun in the sky at the same time. They all had. Slowly, and only in a
few, I watched them struggle to realize that having opposite seasons in
the different hemispheres could not possibly be compatible with their
\quotation{closer to the sun for summer} theory, and that the sun and
the moon in the sky together could not possibly be compatible with their
\quotation{Earth blocks the suns rays} theory of the phases.

To me NSF quite missed the point. They thought they were turning up a
\quotation{science problem,} but there are thousands of science
\quotation{facts} and no scientist knows them all ; we should be
grateful that the Harvard and UCLA students didn't \quotation{know the
answers.} What actually turned up is a kind of \quotation{math problem,}
a thinking and learning problem that is far more serious.

Why more serious ? Because the UCLA students and professors ( and their
Harvard counterparts ) knew something that contradicted the very
theories they were trying to articulate and not one of them could get to
that contradictory knowledge to say, \quotation{Hey, wait a
minute\ldots{}}! In some form, they \quotation{knew} about the opposite
seasons and that they had seen the sun and the moon in the sky at the
same time, but they did not \quotation{know} in any operational sense of
being able to pull it out of their memories when thinking about related
topics. Their \quotation{knowings} were isolated instead of set up to be
colliding steadily with new ideas as they were formed and considered.

What was going on with them--and what similarly goes on with children
every day in school ? To understand this, we have to find out how we
humans are \quotation{naturally} set up to think and learn.

We can get a clue from the Bible. King Solomon was held to be the wisest
man who ever lived and it says why : he knew more than 3000 proverbs !
And proverbs work as follows : if you come home from a trip and your
family is glad to see you, then \quotation{Absence made their hearts
grow fonder.} But if you come home from a trip and they aren't
particularly glad to see you then the reason is\ldots{}what ? That's
right, \quotation{Out of sight, out of mind.} Each proverb exists to
give meaning to a particular situation, and each is recalled on a case
to case basis. If the proverb you use today ( or the play or movie you
see today ) contradicts the one from last week, it is of no moment
because proverbs and stories are evaluated mainly on how good they are
right now, not how they compare to the other proverbs and stories in the
pool.

This way of thinking and giving meaning to one's life and society in
terms of stories and narratives is universal over all cultures, and is
in our basic \quotation{wiring} as human beings. It is part of what we
call \quotation{common sense.} And it is the way most of the college
students that NSF and I talked to had \quotation{learned science}--as
isolated cases, stories that would be retrieved to deal with a similar
situation, not as a system of inter related arguments about what we
think we know and how well we think we know it. Story thinking won out.
Claude Levi-Strauss and Seymour Papert have called this incremental
isolated \quotation{natural} learning \quotation{bricolage}--which means
making something by \quotation{tinkering around.} This is one of the
reasons that engineering predates science by thousands of years ; some
constructions can be accomplished gradually by trial and error without
needing any grand explanations for why things work.

Yet if we look back over the last 400 years to ponder what ideas have
caused the greatest changes in human society and have ushered in our
modern era of democracy, science, technology and health care, it should
be a bit of a shock to realize that none of these is in story form !
Newton's treatise on the laws of motion, the force of gravity, and the
behavior of the planets is set up as a sequence of arguments that
imitate Euclid's books on geometry. All scientific papers since then are
likewise given as special kinds of arguments, not stories. Tom Paine's
Common Sense is a forty page argument about why monarchies are not a
good form of government and why a democracy is likely to be better.
( This was not actually \quotation{common sense,} but
\quotation{uncommon sense} since historically, the movement to democracy
is incredibly rare. ) The Federalist Papers are arguments that support
different parts of the design of the Constitution. And the Constitution
itself is a set of principles for building a very complex dynamic
structure that should last for centuries whose \quotation{parts} ( that
is, us !) come and go and are only somewhat intercooperative. It is most
definitely not a story !

Recent studies have shown that less than 5\letterpercent{} of American
adults ( less than 7\letterpercent{} in the UK ) have learned to think
fluently in these modern nonstory forms. A recent perusal of the top 150
selling books in the US ( as of Sept 15th, thanks to www.usatoday.com )
shows that 80\letterpercent{} are in story form, 15\letterpercent{} are
self help books, 1.5\letterpercent{} could be construed to have some
scientific content, and none were in the form of serious argumentative
essays ( occasionally there appears an extended essay such as Bloom's
The Closing of the American Mind, but none in the top 150 in Sept. ).
And these are percentages for the smallish number of Americans that buy
books at all--remember that a bestseller is around 100,000 books, and a
\quotation{run-away bestseller} is usually no more than 1,000,000 books
in a nation of some 250 millions ! Television, of course, is all about
stories, and finds any other form almost impossible to deal with. Note,
for example, how PBS deals with \quotation{serious subjects}--they are
still given as stories, and at their very best, they function as ads for
books that actually hold the real content.

Now my point here is not to urge that stories be given up. I love to
hear and read them, and I love to see them enacted in the theater. If we
couldn't think \quotation{story} in the theater, all we would see are
actors in front of cardboard scenary supported by various noises from
instruments in the pit. To enjoy theater, we have to give ourselves over
to the narrative, experience actors as ourselves, the symbolic scenary
as a place and mood, and the noises from the pit as stirring music. It
works wonderfully well and we can participate deeply in what it means to
be human via this process. But now consider going to a similar building,
with similar people on a stage uttering similar glorious sentences, all
supported by symbolic scenary and stirring music. Sound like theater ?
But here I am refering to a political rally.

What we are so willing to surrender in theater, we had better hold on to
with both hands here ! Since our whole meaning of life and relationships
with others require us to invest symbols with meaning and to give up
part of ourselves to ideas, we have to get pretty sophisticated to work
both sides of the street : to be tender-minded when our souls can be
lifted, and be tough-minded when someone is trying to take them away
from us. I believe that the main goal of learning is to learn that
discernment, to learn how to make symbols work for us.

But just being able to criticize the kind of story in which one is
embedded is not nearly enough, given that so much of important modern
content, both politically and scientifically, is rendered in forms other
than stories. In order to be completely enfranchised in the 21st
century, it will be very important for children to get fluent in the
three central forms of thinking that are now in use :
\quotation{stories,} \quotation{logical arguments,} and
\quotation{systems dynamics.} The question is \quotation{how ?}

One of the arguments advanced for why it is so difficult to get most
children to learn to think in these new ways is that \quotation{this
kind of thinking is hard to learn.} But it is quite hard to learn to
ride a bike, harder still to shoot baskets, and one of the hardest
things to learn how to do is to hit a baseball consistently. If one
watches children trying to learn these skills, what one sees is that
they fail most of the time, but keep on trying until they learn, usually
over years. This is more like their attitude when learning to walk and
talk than the defeatism so often found in schoolwork. In fact, what
really seems to be the case is that children are willing to go to any
lengths to learn very difficult things and endure almost an endless
succession of \quotation{failures} in the process if they have a sense
that the activity is an integral part of their culture.

Montesorri used this very successfully in her schools. Suzuki has had
similar success in music learning via setting up a musical culture in
which the child is embedded. Television and cultural continuity is very
good at providing an environment that includes athletics and certain
kinds of music and dance, and shows what it means to be highly skilled
at them. An impressively large number of scientists either had a
scientist parent or one who was extremely interested in
science--sometimes just extremely interested in \quotation{learning as a
high calling.} Difficulty is not the real issue here. Belonging to a
culture and building a personal identity are. We could call this
\quotation{rite of passage} motivation.

If we hark back to the less than 5\letterpercent{} estimates for the
percentage of the American population that has learned to think in these
new ways and recall that television is not a good medium to show these
new ways of thinking, this means that most children will have no
embedded cultural experience in these ideas before coming to school. I
don't know what percentage of elementary school teachers have learned to
think in these new ways, but I would guess from personal experience that
it is very similar to that of the population as a whole. This means that
it will be very unlikely for most children to experience these new ways
of thinking at home or at school or through television--especially as
embedded into the general ways of doing and thinking which are so
important to how children assign value to what they are going to try
really hard to learn.

Now something that is very hard to do, and which is not seen by a child
as an important \quotation{rite of passage,} is simply not going to be
focused on with the intensity, stick-to-it-ness and tolerance of failure
that is required to get over the hurdles. One of the great problems with
the way most schools are set up is that the children quickly sense that
most of the stuff they are asked to do is not \quotation{real,}
especially as opposed to optional activities like sports and games, art
and music. They know these are \quotation{real,} and a school has to go
to great lengths to make them artificial enough for the children to lose
interest.

Let me give an analogy to how the \quotation{setting up an environment}
strategy might be dealt with--it is drawn from a learning experience I
had as a child.

Suppose it were music that the nation is concerned about. Our parents
are worried that their children won't succeed in life unless they are
musicians. Our musical test scores are the lowest in the world. After
much hue and cry, Congress comes up with a technological solution :
\quotation{by the year 2000 we will put a piano in every classroom ! But
there are no funds to hire musicians, so we will retrain the existing
teachers for two weeks every summer. That should solve the problem !}
But we know that nothing much will happen here, because as any musician
will tell you, the music is not in the piano--if it were we would have
to let it vote ! What music there is, is inside each and every one of
us.

Now some things will happen with a piano in every classroom. The
children will love to play around with it, and a \quotation{chopstick
culture} is likely to develop. This is \quotation{piano by bricolage.
Some will be encouraged by parents to take lessons, and a few rare
children will decide to take matters into their own hands and find ways
to learn the real thing without any official support. Other kinds of
technologies, such as recordings, support the notion of}music
appreciation.  "  It seems to turn most away from listening, but a few
exceptions may be drawn closer. The problem is that \quotation{music
appreciation} is like the \quotation{appreciation} of
\quotation{science} or \quotation{math} or \quotation{computers,} it
isn't the same as actually learning music, science, math, or computing !

But 50 years ago, I had the experience of growing up in a community that
desired \quotation{real music for all,} and found a way to make it work.
It was a little town in New England that only had 200 students in the
high school, yet had a tradition of having a full band, orchestra and
chorus. This required that almost every child become a fluent musician.
The secret is that every child starts off as a musician in their heart
and each has a voice to sing with. They taught us to sing all the
intervals and sight-read single parts in first grade. In second grade we
sang two parts. In third grade we sang four parts and started to chose
instruments. Talent was not a factor, though of course it did show up.
This was something everyone did, and everyone enjoyed. I did not find
out that this was unusual until I moved away. An important sidelight is
that there was a piano in every classroom and all the teachers could
play a little, though I am sure that at least one of the teachers was
not very musical. What seemed to make it work was that the community had
an excellent musical specialist for the elementary grades who visited
each classroom several times a week. I remember that one teacher didn't
like my phrasing in a song and tried to change it, but the specialist
did like it and encouraged me to see if I could phrase the rest of the
song that way.

The central point to this story is not so much that most of the children
became fluent musicians by the time they got to high school--they did
and had done so for generations--but that as far as I can tell, almost
all still love and make music as adults ( including me ).

We can find this \quotation{create an embedded environment and support
classroom teachers with visiting experts} strategy in a number of
schools today. The Open Charter School of Los Angeles has succeeded in
setting up a \quotation{design culture} in their third grade classrooms
that embeds the children in a year-long exciting and difficult adventure
in the large-scale design of cities. The most successful elementary
school science program I know of is in all of the Pasadena elementary
schools and is organized along the same lines. It was developed by Jim
Bowers and Jerry Pines, two Caltech scientists, and the key is not just
an excellent set of curriculum ideas and approaches, but that the
classroom teachers have to gain some real fluency, and there is
important scaffolding and quality control by expert circuit riders from
the district.

To say it again, children start off loving to learn, and most can learn
anything the culture throws at them. But they are best at learning ideas
that seem to be an integral part of the surrounding culture. Having a
parent or teacher that encourages them to study math and science is not
even close to having one that lives math and science ( or seems to ).
This is the strongest pedagogical strategy I have encountered over more
than 25 years of working with children. Technologies--such as books,
musical instruments, pen and paper, bats and balls--can help, but they
are clearly not enough to get kids over the critical hurdles all by
themselves. On the other hand, literacy, music, art, dance, and sports
can all flourish with little or no supporting technologies at
all--supporting adults are all that are needed.

A good rule of thumb for curriculum design is to aim at being idea
based, not media based. Every good teacher has found this out. Media can
sometimes support the learning of ideas, but often the best solutions
are found by thinking about how the ideas could be taught with no
supporting media at all. Using what children know, can do, and are often
works best. After some good approaches have been found, then there might
be some helpful media ideas as well.

Now let me turn to the dazzling new technologies of computers and
networks for a moment. Perhaps the saddest occasion for me is to be
taken to a computerized classroom and be shown children joyfully using
computers. They are happy, the teachers and administrators are happy,
and their parents are happy. Yet, in most such classrooms, on closer
examination I can see that the children are doing nothing interesting or
growth inducing at all ! This is technology as a kind of junk
food--people love it but there is no nutrition to speak of. At its
worst, it is a kind of \quotation{cargo cult} in which it is thought
that the mere presence of computers will somehow bring learning back to
the classroom. Here, any use of computers at all is a symbol of upward
mobility in the 21st century. With this new kind of \quotation{piano,}
what is missing in most classrooms and homes is any real sense of
whether music is happening or just \quotation{chopsticks.}

I have found that there are many analogies to books and the history of
the printing press that help when trying to understand the computer.
Like books, the computer's ability to represent arbitrary symbols means
that its scope is the full range of human endeavors that can be
expressed in languages. This range extends from the most trivial--such
as astrology, comic books, romance novels, pornography--to the most
profound--such as political, artistic and scientific discussion. The
computer also brings something very new to the party, and that is the
ability to read and write its own symbols, and to do so with blazing
speed. The result is that the computer can also represent dynamic
situations, again with the same range : from \quotation{Saturday Morning
cartoons,} to games and sports, to movies and theater, to simulations of
complex social and scientific theories.

The analogy to a library of books and communication systems is found in
the dynamic networking of millions of computers together in the
Internet. Newly added are that one can use this new kind of library from
anywhere on earth, it is continuously updated, and users can correspond
and even work together on projects without having to be in the same
physical location.

To us, working on these ideas thirty years ago, it felt as though the
next great \quotation{500 year invention} after the printing press was
being born. And for a few percent--very like the few that used the book
to learn, understand, and debate powerful ideas and usher in new ways of
thinking about the world--computers and networks are starting to be that
important. The computer really is the next great thing after the book.
But also as with the book, most are being left behind.

Here is where the analogy to books vs. ~television is most sobering. In
America, printing has failed as a carrier of important ideas for most
Americans. Few get fluent enough in reading to follow and participate in
the powerful ideas of our world. Many are functionally illiterate, and
most who do some reading, read for entertainment at home and for
information on the job ( viz the 95\letterpercent{} of bestsellers as
stories and self-help ). Putting the Federalist Papers on the Internet
will eventually provide free access to all, but to have this great
collection of arguments be slightly more accessable in the 21st century
than it is today in public libraries will make no change in how many
decide to read its difficult but worthwhile prose. Once again we are
face to face with something that \quotation{is hard to learn,} but has
lost its perceived value to Americans--they ask why should they make the
effort to get fluent in reading and understanding such deep content ?

Television has become America's mass medium, and it is a very poor
container for powerful ideas. Television is the greatest
\quotation{teaching machine} ever created. Unfortunately, what it is
best at teaching are not the most important things that need to be
learned. And it is so bad at teaching these most important ideas that it
convinces most viewers that they don't even exist !

Now computers can be television-like, book-like, and \quotation{like
themselves.} Today's commercial trends in educational and home markets
are to make them as television-like as possible. And the weight of the
billions of dollars behind these efforts is likely to be overwhelming.
It is sobering to realize that in 1600, 150 years after the invention of
the printing press, the top two bestsellers in the British Isles were
the Bible and astrology books ! Scientific and political ways of
thinking were just starting to be invented. The real revolutions take a
very long time to appear, because as McLuhan noted, the intial content
and values in a new medium is always taken from old media.

Now one thing that is possible with computers and networks, that could
get around some of the onslaught of \quotation{infobabble,} is the
possibility of making media on the Internet that is \quotation{self
teaching.} Imagine a child or adult just poking around the Internet for
fun and finding something--perhaps about rockets or gene splicing--that
looks intriguing. If it were like an article in an encyclopedia, it
would have to rely on expository writing ( at a level chosen when the
author wrote it ) to convey the ideas. This will wind up being a miss
for most netsurfers, especially given the general low level of reading
fluency today. The computer version of this will be able to find out how
old and how sophisticated is the surfer and instantly tailor a
progression of learning experiences that will have a much higher chance
of introducing each user to the \quotation{good stuff} that underlies
most human knowledge. A very young child would be given different
experiences than older ones--and some of the experiences would try to
teach the child to read and reason better as a byproduct of their
interest. This is a \quotation{Montesorri} approach to how some media
might be organized on the Internet : one's own interests provide the
motivation to journey through an environment that is full of learning
opportunities disguised as toys.

This new kind of \quotation{dynamic media} is possible to make today,
but very hard and expensive. Yet it is the kind of investment that a
whole country should be able to understand and make. I still don't think
it is a real substitute for growing up in a culture that loves learning
and thinking. But in such a culture, such new media would allow everyone
to go much deeper, in more directions, and experience more ways to think
about the world than is possible with the best books today. Without such
a culture, such media is likely to be absolutely necessary to stave off
the fast approaching next Dark Ages.

Schools are very likely the last line of defence in the global
trivialization of knowledge--yet it appears that they have not yet
learned enough about the new technologies and media to make the
important distinctions between formal but meaningless activities with
computers and networks and the fluencies needed for real 21st century
thinking. At their best, schools are research center for finding out
interesting things, and like great research centers, these findings are
best done with colleagues. There will always be a reason to have such
learning centers, but the biggest problem is that most schools today are
not even close to being the kinds of learning centers needed for the
21st century.

Will Rogers once said that its not what you don't know that really hurts
you, but what you think you know ! The best ploy here--for computing,
science, math, literature, the arts, and music--is for schools to be
quite clear that they don't know--they are the blind people trying to
figure out the elephant--and then try to find strategies that will help
gradually reveal the elephant. This is what the top professionals in
their fields do. We find Rudolph Serkin in tears at age 75 accepting the
Beethoven medal, saying \quotation{I don't deserve this} and meaning it.
We find Nobel physicist Richard Feynmann telling undergraduates in his
physics course at Caltech just how much he doesn't understand about
physics, especially in his specialty ! We can't learn to see until we
realize we are blind.

The reason is that understanding--like civilization, happiness, music,
science and a host of other great endeavors--is not a state of being,
but a manner of traveling. And the main goal of helping children learn
is to find ways to show them that great road which has no final
destination, and that manner of traveling in which the journey itself is
the reward.

\stoptext
