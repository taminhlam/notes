\mainlanguage[en]
% Enable hyperlinks
\setupinteraction
  [state=start,
  title={Tomorrow is the day after Doomsday},
  author={John Horton Conway},
  style=,
  color=,
  contrastcolor=]

\setuppagenumbering[location={footer,middle}]
%~ \setupbackend[export=yes]
\setupstructure[state=start,method=auto]

\setupwhitespace[medium]



\setuphead[chapter, section, subsection, subsubsection, subsubsubsection, subsubsubsubsection][number=no]



% \setupexternalfigures[directory={images, /home/user/images}]
%~ \setupexternalfigures[directory={[[foobar]]}]
\setupexternalfigures[directory={doomsday}]


\environment style
\environment colorbrouwer


\starttext
\title[Tomorrow is the day after Doomsday]{Tomorrow is the day after Doomsday}

\placeaside[right,high]{}{
  \startaside[width=.3\textwidth]
    \unhyphenated{
      \startitemize[none, fit][stopper={}]
        \item  John Horton Conway
        \item October 1973
      \stopitemize }
  \stopaside }

by {\bf J. H. Conway}\crlf
from {\em Eureka. October 1973. p. ~28-32.}

Lots of people have produced rules for working out the day of the week
corresponding to any given date. One need merely add components for the
century, year of century, month, and day of month, reduce modulo 7 and
then start counting at the right place. But since the month components
are essentially random numbers most people soon forget the rule. {[}In
the version known as Zeller's congruence the month numbers are produced
by a uniform formula, but this is just complicated enough to be easily
forgotten.{]}

The \underbar{Doomsday rule} is one I worked out last year in an attempt
to overcome these difficulties. In it one computes Doomsday for the
given year and then computes dates in that year relative to Doomsday.
The rule has the additional advantage that when one knows Doomsday for
the year one has in effect the complete calendar for that year at one's
fingertips --- so all the man in the street need do is remember to
update Doomsday at about the time he remembers to put the new year on
his cheques.

\subsection[title={Doomsdays in a given
year},reference={doomsdays-in-a-given-year}]

\startplacetable[location=none]
\startxtable
\startxtablehead[head]
\startxrow
\startxcell  \stopxcell
\startxcell[align=left] DOOMSDAYS \stopxcell
\stopxrow
\stopxtablehead
\startxtablebody[body]
\startxrow
\startxcell Jan \stopxcell
\startxcell[align=left] 31/32 \stopxcell
\stopxrow
\startxrow
\startxcell Feb \stopxcell
\startxcell[align=left] 28/29 \stopxcell
\stopxrow
\startxrow
\startxcell Mar \stopxcell
\startxcell[align=left] 3 + 4 = 7 \stopxcell
\stopxrow
\startxrow
\startxcell Apr \stopxcell
\startxcell[align=left] 4 \stopxcell
\stopxrow
\startxrow
\startxcell May \stopxcell
\startxcell[align=left] 5 + 4 = 9 \stopxcell
\stopxrow
\startxrow
\startxcell June \stopxcell
\startxcell[align=left] 6 \stopxcell
\stopxrow
\startxrow
\startxcell July \stopxcell
\startxcell[align=left] 7 + 4 = 11 \stopxcell
\stopxrow
\startxrow
\startxcell Aug \stopxcell
\startxcell[align=left] 8 \stopxcell
\stopxrow
\startxrow
\startxcell Sep \stopxcell
\startxcell[align=left] 9 − 4 = 5 \stopxcell
\stopxrow
\startxrow
\startxcell Oct \stopxcell
\startxcell[align=left] 10 \stopxcell
\stopxrow
\startxrow
\startxcell Nov \stopxcell
\startxcell[align=left] 11 − 4 = 7 \stopxcell
\stopxrow
\stopxtablebody
\startxtablefoot[foot]
\startxrow
\startxcell Dec \stopxcell
\startxcell[align=left] 12 \stopxcell
\stopxrow
\stopxtablefoot
\stopxtable
\stopplacetable

\underbar{Doomsday} for a given year is defined to be the day of the
week on which the last day of February falls. For 1973,
\underbar{Doomsday is Wednesday}: The displayed table shows how to find
a Doomsday in any given month. In February we see that a Doomsday is Feb
28 or 29 according as the year is ordinary or leap. In January, a
Doomsday is Jan 31 or 32 in the same circumstances, ( Of course Jan 32 =
Feb 1, but we should think of it as Jan 32. )

Otherwise 2 Doomsday in the nth month is the nth day, if n is even, and
the ( n ±4 )th day, if n is odd. The sign is + for long odd months ( 31
days ), and − for short ones ( 30 days ), and it is fairly easy to
remember than the only short odd months are September and November.

Summary : \quote{Last} in Jan and Feb, otherwise nth in even months,
( n ±4 )th in odd ones.

By adding and subtracting 7s we can find other Doomsdays in these
months, and then by the
\quotation{last-friday-was-the-twentyfirst-so-today-is-the-twentyfifth}
technique we can locate any particular date.

\subsection[title={Examples},reference={examples}]

\underbar{August 19 1973}? August is the 8th month, so that August 8th,
and therefore August 22nd are Doomsdays ( Wednesdays in 1973 ), so
August 19th is a Sunday.

\underbar{September 24th}? September is the 9th month, and is short, so
September 9 − 4 and September 26 are Doomsdays, so September 24th 1973
is a Monday.

II you do things exactly this way you will gradually remember more and
more Doomsdays throughout the year. Don't say things like Sep 5 =
Doomsday, 24 − 5 = 19 ≡ −2 ( mod 7 ), so Sep 24 = Doomsday − 2. This
kind of calculation is prone to sign errors and does not help you to
accumulate more mental Doomsdays.

\subsection[title={Doomsdays for the century
years.},reference={doomsdays-for-the-century-years.}]

One first needs to know Doomsdays for the century years. All the
practical man need know is that
\underbar{Doomsday for 1900 was a Wednesday}. ( We say simply
\quotation{1900 was a Wednesday}. ) However, we assert that in the
Julian system ( which was used in England before September 1752 ) the
years 0, 700, 1400, 2100, \ldots{} are \underbar{Sundays} ( they were
more Godly then !), and that each century after one of these retards
Doomsday by 1 day. In the Gregorian system ( after September 1752 ) 0,
400, 800, 1200, 1600, \ldots{} are \underbar{Tuesdays}, and each century
after the most recent of these retards Doomsday by 2 days. In
particular, 1900 = 3 centuries past 1600 = Tuesday − 6 = Wednesday, as
we asserted.

\subsection[title={Doomsdays for years in a given
century.},reference={doomsdays-for-years-in-a-given-century.}]

Each ordinary year has its Doomsday 1 day later than the previous year,
and each leap year 2 days later. It follows that within any given
century a dozen years advances Doomsday by 12 + 3 = 15 days = 1 day.
(\quotation{A dozen years is but a day.}) So we add to the Doomsday for
the century year the number of dozens of years thereafter, the
remainder, and the number of fours in the remainder. It is easiest to
say these numbers aloud, as in the examples. Remember, 1900 = Wednesday.

1946 ? We have 46 = 3 dozen and 10, so we say

\startblockquote
\quotation{1946 = Wednesday, 3 dozen, 10, and 2 = Thursday.}\crlf
( Thursday being found by adding 3, 10, and 2 to Wednesday. )
\stopblockquote

1973 ? Wednesday, 6 dozen and 1 = Wednesday, as we know.\crlf
1990 ? Wednesday, 8 dozen\footnote{? ! sic.}, 5, and 1 = Thursday.

\underbar{1752 September 2 (Julian)?}

1400 = Sunday, so 1700 = Sunday − 3 = Thursday, so\crlf
1752 = Thursday, 4 dozen, 4, and 1 = Saturday, so\crlf
September 5 would be Saturday, and September 2 = Wednesday.

\underbar{1752 September 14 (Gregorian)?}

1600 = Tuesday, so 1700 = Tuesday − 2 = Sunday, so\crlf
1752 = Sunday, 4 dozen, 4, and 1 = Tuesday, so\crlf
September 5 and 12 would be Tuesdays, so September 14 = Thursday.

In fact in this country Wednesday September 2 and Thursday September 14
1752 were consecutive days, since this was the year we changed from
Julian to Gregorian.

\subsection[title={\underbar{Note on changes in the calendar. }},reference={note-on-changes-in-the-calendar.}]

The Julian system in which every fourth year has an extra day was
introduced by Julius Caesar on the advice of the astronomer Sosigenes.
For some time the calendar had been at the mercy of Roman officials who
more or less arranged things to suit themselves, and there had, for
instance, even been one year with a month of 45 days. Sosigenes
recommended a regular alternation of 30 and 31 day months which would
have been adhered to had not both Julius and Augustus Caesar needed the
months named after them to have 31 days, which was achieved by breaking
the regular alternation and shortening February still more.

The Gregorian system, in which years divisible by 100 but not 400 are
not leap years was introduced by Pope Gregory XII. Roman catholic
countries changed in 1582, but protestant countries resisted this piece
of popery for several hundred years, and then changed at various times.
Some eastern European countries changed only this century, Sweden
managed the change most elegantly, by simply omitting all leap years
between 1800 and 1840 inclusive. So in working out dates for the
intervening Period, one must be sure where one's problem originated.

The reason for the change was of course that the astronomical year has
365.2422 days rather than 365¼, and the inaccuracy had gradually
accumulated until it was 10 or 11 days. There is still a residual
inaccuracy which many people have remarked would be partly cured by
making the years divisible by 4000 not leap years, but with any luck the
whole ungainly system will be dead by then !)

Another annoyance is that the conventional starting date for the year
has not always been January 1 ( as we have supposed in the Doomsday
rule ). A number of different dates have been used at various times,
even in this country. Jan 1 and Dec 25 ( of what we should call the
previous year ) were both used at about the end of the first millennium,
but March 25 then became more or less universal. So for instance March
24 1583 and March 25 1584 were consecutive days.

This convention for starting the year is the \underbar{Old Style}, the
January 1 convention being the \underbar{New Style}. Unfortunately these
terms are often used incorrectly to refer to the Julian and Gregorian
systems, since in fact the Act of Parliament establishing the Gregorian
system in England also finally decreed that the New Style was henceforth
to be used for all legal purposes.

In fact the change from Old to New style dating had been accomplished
long before. From about 1600 to 1700 opinion had gradually hardened in
favour of the New Style. In the changeover period we usually find the
double dating convention --- thus February 14 166 ⅚ denotes the date
which would be Feb 14 1665 ( Old Style ), and Feb 14 1666 ( New Style ).
The situation is further complicated by the fact that dates in history
hooks have often been transposed into New Style even when they refer to
periods when Old Style was the only one in use. Also, dates in English
history books for the deaths of French kings ( say ) might be in either
the Julian or Gregorian system for the period between 1582 and 1752,
according as the original source was English or French.

The fact that for various financial purposes the year starts on April 5
is an interesting consequence of the various changes. Originally it
started in the first day ( March 25 ) of the calendar year. When New
Style was adopted, this remained the start of the financial year,
although no longer the start of the calendar year. When we changed from
Julian to Gregorian, this became April 5, since obviously no one was to
pay a full year's interest on a year that was eleven days short !

So apply the method for historical dates with some caution. But the rule
really comes Into its own for dates within any given year. Throw your
calendar away after a quick glance to find the Doomsday, for when you
know Doomsday you will know it all ! But when you impress your friends
with the Doomsday rule, remember to give credit where credit is due ! I
do this now by noting that I found the Doomsday rule by simplifying
( almost beyond recognition !) a rule given by Lewis Carrol in
\underbar{Nature}, 1872.

\stoptext
