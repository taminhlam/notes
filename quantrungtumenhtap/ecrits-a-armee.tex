\mainlanguage[fr]
% Enable hyperlinks
\setupinteraction
  [state=start,
  title={Écrits à l'Armée},
  author={Nguyen Trai},
  style=,
  color=,
  contrastcolor=]

\setuppagenumbering[location={footer,middle}]
%~ \setupbackend[export=yes]
\setupstructure[state=start,method=auto]

\setupwhitespace[medium]



\setuphead[chapter, section, subsection, subsubsection, subsubsubsection, subsubsubsubsection][number=no]



% \setupexternalfigures[directory={images, /home/user/images}]
%~ \setupexternalfigures[directory={[[foobar]]}]
\setupexternalfigures[directory={quantrungtumenhtap}]


\environment style
\environment colorbrouwer


\starttext
\title[Écrits à l'Armée]{Écrits à l'Armée}

\placeaside[right,high]{}{
  \startaside[width=.3\textwidth]
    \unhyphenated{
      \startitemize[none, fit][stopper={}]
        \item  Nguyen Trai
      \stopitemize }
  \stopaside }

\startsectionlevel[title={Écrits à l'Armée},reference={écrits-à-larmée}]

Le recueil, {\em Écrits à l'armée} ({\em Quan Trung Tu Menh Tu})
comprend des lettres addressées aux generaux ennemis, des ordres,
proclamations addressees aux combattants, rédigées par Nguyên Trai au
nom du roi pendant la guerre de liberation contre les Ming.

\startsectionlevel[title={RÉPONSE À PHUONG
CHINH},reference={réponse-à-phuong-chinh}]

{\em Les résistants avaient installé leur base dans l'ouest montagneux
de la province de Thanh Hoa. Le general ennemi Phuong Chinh leur avait
envoyé une lettre d'accusation. Nguyen Trai lui répond.}

Pirate Phuong Chinh, je te le dis. L'art militaire part des grandes
vertus d'humanité et de justice, pour aboutir au courage et à
l'intelligence. Toi et les tiens ne savez que duper, assassiner des
innocents, acculer les gens à la mort, sans aucune pitié. Le ciel ne
peut tolérer de pareils forfaits, hommes et génies les maudissent. C'est
pourquoi, la défaite vous suit à longueur d'année, au cours de chaque
expédition, Au lieu de reconnaitre vos crimes et de réparer vos fautes,
vous remuez la fange ; ne sera-t-il pas trop tard pour vous en repentir.
Or, les eaux printanières montent, les miasmes s'accumulent, vous ne
sauriez tenir bien longtemps. Et toi, général, avec tes armées en main,
tu n'oses plus avancer, condamnant tes hommes à mourir de toutes les
pestilences, A qui la faute ? La théorie militaire veut que
\quotation{celui qui possède la vertu d'humanité peut s'appuyer sur la
faiblesse pour subjuguer la force, celui qu'anime la justice peut
opposer un petitcnombre au plus grand nombre}. Si tu veux vraiment te
battre, avance, finissons-en une bonne fois, ne tergiverse plus, il faut
mettre enfin un terme aux misères des deux armées.

\stopsectionlevel

\startsectionlevel[title={NOUVELLE RÉPONSE À PHUONG
CHINH},reference={nouvelle-réponse-à-phuong-chinh}]

{\em Phuong Chinh a répond à la premiere lettre en incitant les
résistants à quitter la montagne pour engager le combat dans la plaine.
Nguyen Trai lui répond.}

Pirate Phuong Chinh, je te le dis. De tous temps, pour un bon général,
il n'y a pas de terrains accidentés ou favorables, de champs de bataille
faciles ou difficiles. La victoire ou la défaite dépend des capacité de
ceux qui commandent, et nullement du terrain. Quand deux armées
s'engagent sur un terrain accidenté, c'est comme deux tigres qui ont
choisi de se battre dans une vallée encaissée, le meilleur vaincra.

La configuration du terrain ne joue pas toujours dans un seul sens,
comme la situation réciproque de deux armées. Qu'est-ce qui permet de
préjuger si un terrain est favorable ou non ? Si tu n'es pas décidé à
t'enfuir, fais avancer ton armée, engage résolument le combat.

\stopsectionlevel

\startsectionlevel[title={NOUVELLE LETTRE À PHUONG
CHINH},reference={nouvelle-lettre-à-phuong-chinh}]

{\em Les résistants ont cerné l'ennemi dans la citadelle de Nghê An, au
milieu de l'année 1426.}

Pirate Phuong Chinh, je te le dis. Il est reconnu qu'un bon général
choisit l'humanité et la justice et non les ruses et les stratagèmes.
Toi et les tiens, vous n'avez guère ruses et stratagèmes, comment parler
d'humanité et de justice. Il fut un temps où tu me narguais de m'abriter
derrière les montagnes, hésitant comme un rat, n'osant engager le combat
en plaine. Nous voici cernant la citadelle de Nghê An, de partout le
terrain est propice au combat. Oses-tu encore parler des montagnes et
des forêts, toi qui reste tapi au fond d'une citadelle bien close, comme
une vieille paralytique ? Je crains de ne pouvoir vous épargner honte de
vous envoyer des cache-seins\footnote{Allusions a une histoire chinoise
  de la période des Trois Royaumes (220-288) : le general Tu Ma Y
  s'étant entermé dans une citadelle, refusant de se battre, son ennemi
  Gia Cat Luong lui envoya une echarpe et un corsage voulant signifier
  ainsi qu'il Le considérait comme une femme.}.

Qui nourrit de grands desseins doit s'appuyer sur les principes
d'humanité et de justice ; les grandes œuvres commencent toujours par
des actes justes et humains. C'est seulement à force d'humanité et de
justice qu'on mène à bien de grandes entreprises. Profitant des fautes
de la dynastie Hô\footnote{La dynastie Hô ayant usurpé le trône, les
  Ming en profitèrent envahir le pays.} ton gouvernement, sous pretexte
de \quotation{sauver le peuple de chatier les coupables}, a usé de
violence, agressé, envahi notre pays, pillé notre natiom, lui extorquant
force impôts et trésors, lui infligeant des peines attroces. Le menu
peuple dans les hameaux les plus reculés ne connait plus la paix. Est-ce
là humanité et justice ?

Voici que ton pays s'est attiré la haine du peuple et la colère des
genies. Les grands deuils se succèdent. Au lieu de faire l'examen de
Conscience, tu fomentes la guerre agresses un pays lointain, exposes tes
hommes à la mort accules ton peuple au malheur. J'ai bien peur que le
danger pour vous ne se trouve point au-delà des frontières, mais dans
votre propre pays\footnote{Allusiom aux troubles intérieurs qui
  agitaient la Chine des Ming.}.

\stopsectionlevel

\startsectionlevel[title={NOUVELLE LETTRE A VUONG THONG ET SON
THO},reference={nouvelle-lettre-a-vuong-thong-et-son-tho}]

{\em Les pourparlers de paix avaient commencé. Mais les généraux Ming
continuaient leurs préparatifs de guerre.}

Il est dit : Tenir sa parole est le bien le plus précieux pour un État,
Que peut faire hélas, un homme qui ne possède pas cette vertu ? J'ai
reçu tout dernièrement votre messager et vos propositions de paix, et de
mon côté, j'ai agi en conséquence, Et que vois-je ? Dans votre
citadelle, l'on continue à creuser tranchées et chausse-trapes, à ériger
palissades et remparts, à saccager nos antiquités pour faire des armes
et des fusées, Alors, allez-vous retirer vos troupes ? ou
nourrissez-vous le dessein de camper indéfiniment dans cette citadelle ?
Je ne vois pas clair dans vos intentions, il est écrit dans les
classiques : sans sincérité, rien ne va, Si Je cœur n'est pas sincère
tout acte est empreint de mensonge. Si vraiment vous ne voulez pas
renier vos engagements, que vos actes le disent en toute clarté. Si vous
voulez retirer vos troupes, retirez-les. Si vous voulez vous cramponner
ici, faites-le. À quoi bon Parler de paix en préparant le contraire ? Ne
faites pas preuve de tant d'inconséquence. Malgré son ignorance, le menu
peuple n'en est pas moins lucide. Tout bête que je suis, je ne m'en
rappelle pas moins le précepte de Confucius : (pour connaitre un homme)
scrute ses actes, examine ses motifs, vois s'il se réjouit de ce qu'il
fait. Et ce que font les autres, de vrai, de faux, ne m'échappe guère.
Ce message est loin de dire tout ce que je pense.

\stopsectionlevel

\startsectionlevel[title={LETTRE À VUONG
THONG},reference={lettre-à-vuong-thong}]

{\em Le général Vuong Thong, assiégé dans la citadelle de Dong Quan,
refusait de déposer les armes.}

J'ai souvent entendu dire qu'à la franchise, les hommes répondent
toujours par la franchise. La véritable sincérité remue jusqu'au Ciel et
ses génies, à plus forte raison les hommes. Obéissant aux ordres de
votre Empereur pour agir au-delà des frontières, vous auriez dù faire
preuve de franchise, mais vous croyant habile, et me prenant pour un
ignorant, vous avez usé de tromperies et d'artifices, Vous parlez de
paix, mais projetez le contraire ; vous aviez promis de retirer vos
troupes dès que vous auriez adressé votre supplique à votre Empereur,
mais vous Voici renforçant vos tranchées et vos remparts. Est-ce là
franchise ou fourberie ?

Jadis, un grand dignitaire en mission au-delà des frontières avait le
droit de décision en toutes choses. Vous, ayant grade de généralissime,
et qui possédez tous les classiques, à qui la passation des droits a été
faite en grande pompe, vous ne sauriez arguer que vous devez attendre
des ordres pour la moindre chose ? Vous savez bien que l'art militaire
exige de la rapidité, que ses mécanismes s'ouvrent et se ferment à la
vitesse des chars qui roulent, des nuages qui filent, que la situation
en un instant peut virer du chaud au froid, pourquoi alors prendre
conseil aupres d'un subalterne aussie fourbe que Ma Ky, d'un officier en
déroute comme Phuong Chinh, pour tergiverser sans fin et ajourner toute
décision ? Vous avez donné des ordres pour que soient retirées les
garnisons locales, maintenant vous prétextez de l'exiguité de la
citadelle pour dire que ces garnisons ne seront renvoyées qu'aprés
l'évacuation de la citadelle. Les troupes de Zien, de Nghe sont là, mais
votre parole s'évanouit comme fumée au vent. Non seulement vous m'avez
induit en erreur, mais avez encore trompé sept à huit mille hommes de
ces garnisons. Par déférence pour votre gouvernement, et prenant en
pitié le sort de ces milliers de soldats, j'ai rigoureusement interdit à
nos troupes de leur faire le moindre mal. Et vous, suivant le conseil de
vulgaire fripons, vous cherchez à me nuire, du coup à nuire à tant
d'autres. À vrai dire, les chevaux originaires du Nord hennissent chaque
fois que souffle la bise, les oiseaux migrateurs des contrées
méridionales cherchent toujours une branche orientée vers le Sud. Quel
est l'homme qui me pense pas à sa terre natale ? Maintenant vos projets
tombent à l'eau, des milliers d'hommes des garnisons nourrissent à votre
égard une haine sans bornes, grincent des dents, serrent leur poing et
jurent de ne plus vous revoir en ce monde. Ils m'ont tous demandé d'en
finir dans un combat décisif. Si vous tenez à respecter vos engagements,
retirez vos troupes. Moi, je vous remettrai les gommes et les chevaux
capturés dans diverses garnisons. Autrement, ces milliers de soldats qui
vous en veulent à mort, et mes trois cent mille hommes, je devrai les
faire monter jusqu'aux abords de vos remparts. A vous de prendre une
décisions. Je tremble en pensant à la mienne.

Ce mot n'arrive point à dire tout ce que je pense.

\stopsectionlevel

\startsectionlevel[title={NOUVELLE LETTRE À VUONG
THONG},reference={nouvelle-lettre-à-vuong-thong}]

L'on m'a conté qu'un homme ayant lâché ses chiens et ses faucons dans
une forêt, jeté ses filets dans un lac, avait déclaré: \quotation{Je ne
chasse, ni ne pêche}. Il avait beau avoir une langue de sept pouces, il
ne trompait personne. Il lui aurait suffi pourtant de rentrer ses chiens
et ses faucons, de retirer ses filets et tous l'auraient cru. Vous même
m'aviez maintes fois écrit que sur l'ordre de l'Empereur Yong
Lo\footnote{L'empereur Yong Lo (1403-1424) des Ming. Avant d'envahir
  notre pays, il fit proclamer dans l'intention de tromper notre peuple,
  que ses troupes anéantiraient la dynastie usurpatrice des Hô pour
  rétablir la dynastie légitime des Tran.}, l'expédition du Giao
Chi\footnote{Gia Chi : un des anciens noms du Vietnam.} ne cherchait
qu'à assurer la sucession de Tran, et que ma supplique d'investiture à
l'Empereur une fois adressée, vos troupes se retireraient, Au début,
nous étions tous pleins de joie et de confiance. Puis nous avons vu que
la citadelle de plus en plus s'entourait de remparts, que s'activaient
les préparatifs militaires. Nous avons commencé à douter, à craindre.
Moi qui ai bénéficié de vos bienfaits immenses comme le Ciel et la
Terre, je commence à m'en repentir, que dire des autres ? Vous prétendez
ne chasser ni pêcher, mais vous n'avez point rentré vos chiens et vos
faucons, ni retiré vos filets pour inspirer confiance.

Dernièrement un homme de Khau On, dévoré de douleur, a trouvé le message
que vous avez rédigé le 10e jour du 12e mois de l'an 1 du règne Tuyen
Duc, et confié au mandarin autochtonee Vu Nhan, ainsi que celui du 16e
jour du même mois, confié à un autochtone, Tu Thanh, pour être adressé à
l'Empereur.

A les lire, je m'aperçois que vos vertus œuvrent comme le Ciel nourrit
les êtres, sans que ces derniers en soient avertis. Dans le premier
message, vous me reprochez de \quotation{manquer de piété envers le
Ciel, la Terre, et les parents}. N'ayant aucune conscience de mes
péchés, je ne savais que trembler de peur, mais croyais toujours en
votre générosité. Dans ces messages, j'ai aussi lu \quotation{qu'il ne
fallait point harasser le peuple pour un coin de terre}. Quelles Justes
Paroles ! SI chacun s'inspirait de ces nobles sentiments, le monde
aurait connu la paix. Et pourtant, vos gouverneurs de province, vos
dignitaires à la Cour, vos ministres n'en finissent plus de tenir
conseil, n'arrivent pas à s'accorder entassent pl : ne sais pourquoi ces
Excellences tergiversent et jusqu'à quand prolongent leurs
délibérations ! Si, affirmant votre sincérité, et comme vous l'avez dit
dans une de vos lettres, vous avez reçu blanc-seing pour agir, vous avez
le droit de retirer les troupes avant d'en recevoir l'ordre, prenez-en
donc la décision. Vous dénouerez les haines, sauverez les êtres humains,
couvrez le monde de vos bienfaits, aiguillerez votre peuple dans la
juste voie, léguerez un grand nom à la postérité.

À quoi servent quelques dizaines de milliers de renforts ? Réfléchissez
bien, le mieux est de rentrer vos chiens et vos faucons, de retirer vos
filets. Si, par bonheur, vous tenez à vos engagements, de mon côté, je
réparerai routes et ponts fournirai les vivres, me tiendrai tout prèt
pour faciliter le retour de vos troupes de Nghê An, de Thuan Hoa, de Tan
Binh, de Tien Ve, et ne toucherai pas à un seul de leurs cheveux. Je
suis toujours à vos ordres. Les doutes réciproques s'évanouiront. Le
Ciel, La Terre en sont témoins, et que les génies exterminent celui qui
manquera à sa parole. Puissiez-vous accorder votre attention à mon
message.

\stopsectionlevel

\startsectionlevel[title={LETTRE AUX RESPONSABLES DE LA CITADELLE DIEU
ZIEU\footnote{Dieu Zieu : dans l'actuel district de Gia Lam (banlieue de
  Hanoi). Les responsables sont des traîtres au service des Ming.}},reference={lettre-aux-responsables-de-la-citadelle-dieu-zieu7}]

Les anciens disaient : \quotation{Les corbeaux retournent toujour au
nid, les renards en mourant se tournent toujours vers la colline
natale}. Ainsi font les bêtes, les hommes seraient-ils moins sensés ?
Vous êtes gens de notre pays, de notre peule à la civilisation
millénaire. Quand les Hô faillirent à leurs obligations, et que l'ennemi
envahit notre pays, certains d'entre vous furent retenus à la Cour de
l'occupant, d'autres forcés d'accepter des foncrions de valet, je sais
bien que ce n'était point de gaieté de coeur.

Le Maître Supréme, Prenant notre peuple en compassion s'est servi de ma
personne pour executer les volontés du Ciel. J'ai donc assumé la charge
de \quotation{Duc Protecteur du Pays, Grand Maitre gouvernant au nom du
Ciel} afin de sauver le peuple, châtier les coupables et restaurer la
nation. Là où nos troupes pénètrent, la juste cause retentit, la
population entière, y compris les mères portant leurs enfants sur le
dos, se hâtent de se joindre à nous.

Vous, il vous suffit de vous repentir, de renoncer à la trahison, de
rallier le droit chemin, de Vous rendre ou de servir comme agent
camouflé dans les organes ennemis. Non seulement vous pouvez laver la
honte du passé, mais je ne manquerai pas non plus de penser à vous au
lendemain de la victoire. Je tiendrai ma promesse.

Mais si vous tenez à vos fonctions de valet, et si vous vous opposez à
l'armée royale, le Châtiment qui vous attend à la chute de la citadelle
sera certainement plus sévère que Le les ennemis.

\stopsectionlevel

\startsectionlevel[title={LETTRE À VUONG
THONG},reference={lettre-à-vuong-thong-1}]

Je vous ai adressé un message et je n'ai pas eu l'honneur d'avoir de
réponse. Je viens de vous envoyer un messager et vous n'avez pas daigné
le recevoir. Pourtant, vous aviez bien dit que \quotation{les paroles et
les actes ne doivent pas se contredire}. Qu'est-il advenu de cette belle
maxime ? Représentant un petit pays vis-à-vis d'une grande puissance, je
n'ai jamais manifesté que crainte et respect. Suivant vos propres
paroles, je n'ai pas osé laissé se rompre les liens qui ont été tissés
entre nous, et nonobstant toutes les difficultés, j'ai toujours continué
à vous adresser message sur message. Les résultats n'ont nullement
répondu à mon attente.

Serait-ce la situation qui en serait responsable ? Je me suis permis de
me mettre à votre place et j'ai pensé que le mieux serait de retirer vos
troupes, pour faire cesse cette guerre qui, n'en finissant plus, accable
les deux pays, pour éviter à votre peuple le malheur de se lancer dans
une guerre injuste. Vous auriez ainsi l'insigne mérite d'avoir restauré
un pays ruiné, une dynastie déchue, d'avoir fait preuve d'humanité, ne
considérant que l'intérêt du peuple et sacrifiant le vôtre. Vous auriez
bien mérité de la confiance que la cour impériale mettait en vous sans
Outrepasser les obligations d'un général en mission au-delà des
frontières. Quoi de plus beau ! Les annales garderaient à jamais votre
nom.

Si vous choisisse de suivre l'ornière des Han et des Tang, courant après
les grandes œuvres et les grandes victoires, pourquoi ne pas conduire à
la bataille des combattants d'une juste cause pour sauver le peuple et
châtier les coupables ? Vous avez abandonné tout cela, vous vous
évertuez à creuser des fossés, à élever des remparts ; jour après jour,
vos troupes à l'affût derrière les murs, cherchent la moindre occasion
pour aller rafler un peu de bois ou quelques touffes
d'herbes\footnote{Les soldats Ming assiégés manquant de vivres.}.
Pourquoi toutes ces misères ?

Vous pensez que fosses et remparts peuvent vous protéger, j'ai bien peur
qu'une source d'eau lointaine ne puisse éteindre un incendie tout
proche\footnote{Les renforts venant de Chine sont aléatoires.}. Vou
pensez disposer dans la citadelle de bonnes troupes, et pouvoir engager
une bataille décisive. Rappelez-vous, quand j'étais encore à Kha Lam et
Tra Lan, Phuong Chinh opposait des dizaines de milliers d'hommes
entraînés à mes quelques centaines de combattants ; pourtant unis comme
fils et pères, partout nous avons vaincu, et avancions comme l'on fend
un bambou.

Maintenant qu'ayant mobilisé les hommes des provinces Zien, Nghe, Thanh,
Tan, Thuan, Dong Do, je dispose de plusieurs centaines de milliers de
troupes aguerries : l'issue est facilement prévisible. Grandeur et
décadence des royaumes sont questions de mandats célestes ; force ou
faiblesse d'une armée ne dépend nullement du nombre. Vous vous obstinez
à raisonner sur l'exemple des Hô. Or il n'y a rien de comparable entre
la situation d'hier et celle d'aujourd'hui.

Les Hô cherchaient à tromper le Ciel, à nuire au peuple. Nous, nous
respectons la volonté du Ciel et sommes avec le peuple. Accord et
désaccord avec la volonté du Ciel et du peuple, telle est la premiere
différence. Les soldats des Hô étaient un million, mais étaient déchirés
par un million d'opinions diverses ; mes hommes ne sont que quelques
centaines de milliers, mais tous lutte d'un même Cœur. Telle est la
seconde différence.

Si, passant outre aux mauvais conseils, vous décidez de rentrer au pays,
que Son Excellence Son et des messagers de confiance passent le fleuve
pour engager des pourparlers, je retirerai immediatement mes troupes
vers Thach That, Thanh Dam, Khoai Chau afin de vous preparer le chemin.
Toute autre solution ne peut que mener à une impasse.

\stopsectionlevel

\startsectionlevel[title={NOUVELLE LETTRE A VUONG
THONG},reference={nouvelle-lettre-a-vuong-thong}]

À bien réfléchir, l'art militaire est une affaire de temps et de
\quotation{situation}. Quand on est en accord avec le \quotation{temps}
et qu'on se trouve dans une bonne \quotation{situation}, ce qui parait
perdu peut subsister, de petites forces agissent à l'égal de grands
moyens. Quand on est en discordance avec le \quotation{temps} et qu'on
se trouve dans une mauvaise situation, la force devient faiblesse, la
stabilité, péril. Les choses peuvent changer en un tournemain.

Vous ignorez tout de ces questions de \quotation{temps} et de
\quotation{situation}, et cherchez à vous camoufler derrière des
duperies et des mensonges, n'est-ce pas faire montre de médiocrité et
d'incapacité? à quoi bon discuter d'affaires militaires ?

Vous avez la fourberie de parler de paix, tout en renforçant vos
remparts et fossés, attendant la venue de renforts, cachant vos
véritables sentiments, actes et paroles se contredisant : pensez-vous
nous inspirer confiance de cette façon ?

Les anciens disaient : \quotation{Le cœur des autres, nous pouvons le
jauger}. C'est bien clair, n'est-ce pas ? Jadis les Ts'in ont annexé six
royaumes, régné sur quatre mers sans se soucier d'entretenir leurs
vertus. Ils ont perdu et leur vie et leur empire. Votre force n'égale
point celle des Ts'in, mais vous les surpassez en cruauté. En moins d'un
an, deux souverains sont morts l'un après l'autre\footnote{Les rois
  Thanh To et Nhan Tong des Ming morts en 1425.}. C'est bien le Ciel qui
a jugé et non les hommes.

Au nord de l'empire, les Thien Nguyen campent toujours à l'intérieur,
vous n'arrivez guêre à réprimer les troubles qui agitent la région de
Tam Chau, Giang Ta, et vous projetez de mettre la main sur un autre
pays ! Vous n'avez rien compris à la situation, et battus, vous cherchez
à nous effrayer avec l'ombre de Truong Phu. Est-ce là agir en grand
homme ? Dites plutôt en femmelette ! Où en sont les choses aujourd'hui ?
même si votre Empereur venait commander vos troupes, il ne pourrait que
les conduire à la mort, un Truong Phu ne pourrait tout au plus que
livrer sa propre personne, à quoi bon en parler ?

Jadis Han Chieu Liet n'était que descendant d'une lignée cadette ;
pourtant Khong Minh est arrivé à restaurer son royaume. Aujourd'hui que
les descendants de la famille royale des Tran bénéficient du mandat du
Ciel, du soutien du peuple, comment vous les Ngo pensez-Vous pouvoir
annéxer notre royaume ?

\ldots{}

Vous êtes à bout de force et de ruses, vos soldats sont épuisés. Vous
manquez de vivres, les renforts n'arrivent pas, vous vous accrochez à un
bout de terrain, bivouaquez dans une citadelle isolée, ne cragnez-vous
pas que votre situation soit celle d'un poisson déjà sur le billot, d'un
mouton déjà à l'abattoir ? Et vous cherchez à duper notre peuple et le
corrompre. Regardez vos sujets fidèles, et les combattants héroïques.
Même aux moments les plus sombres, quand il fallait coucher sur des
épines, goûter du fiel, ils ne voulaient s'écarter de leur droit chemin,
comment imaginer qu'ils se laissent séduire par vos propositions
indécentes ?

J'ai peur que nos compatriotes enfermés dans la citadelle ne se tournent
vers leur Ancien Roi, que vos hommes venus de loin ne puissent plus
endurer autant de misères ; ceux qui s'opposent à vos desseins se
rendront en masse, et comme Truong Phi, La Bo, vous serez nécessairement
liquidé par vos propres subordonnés. Aujourd'hui, dans toutes vos
garnisons, tous, depuis les commandants jusqu'aux soldats, sont excédés
par vos tromperies et me poussent à raser toutes les citadelles.
Certains s'étant enfuis hors des remparts m'ont dénoncé tous vos
préparatifs, confection chars, d'échelles, mise en place des engins de
guerre. Les vôtres réduits aux dernières extrémités finiront par
s'entre-tuer, nos troupes n'auront même pas à intervenir. En me mettant
à votre place, je vous trouve six raisons de défaite :

\startenumerate[n,packed][stopper=.]
\item
  Les crues s'accentuent, vos remparts tombent en morceaux, les vivres
  manquent, les chevaux meurent comme des mouches, vos hommes sont
  malades.
\item
  Jadis, Duong Thai Tong capture Kien Duc, et The Sung fut obligé de
  capituler\footnote{Kien Duc qui vint avec des renforts pour sauver The
    Sung assiégé fut capturé par Duong Thai Tong.}. Aujourd'hui, toutes
  nos passes sont gardées par nos hommes et nos éléphants, vos renforts
  courent au-devant d'une défaite certaine. Les renforts défaits, vous,
  vous serez nécessairement capturés.
\item
  Vos troupes d'élite, vos meilleurs chevaux sont concentrés aux confins
  septentrionaux de l'Empire, face aux Nguyên et personne n'a le loisir
  de s'occuper du front du Sud.
\item
  Votre gouvernement ne cesse de fomenter des guerres, de lancer des
  expéditions, accablant le peuple qui s'agite et se désespère.
\item
  Chez vous règne un Empereur trop jeune, des ministres fourbes font la
  loi, des conflits fratricides, des troubles incessants déchirent la
  Cour.
\item
  J'ai appelé nos hommes au nom d'une juste cause, et d'un seul cœur,
  nous luttons avec tout notre héroisme ; chaque jour, nous nous
  endurcissons, chaque jour nous fourbissons nos armes. Nos hommes
  labourent tout en faisant la guerre, tandis que la défaite démobilise
  vos soldats assiégés, et à bout de souflle.
\stopenumerate

Et vous vous cramponnez à cette citadelle minuscule dans l'attente de
cette défaite inéluctable, je ne puis que le regretter pour vous ! Un
ancien dicton nous apprend \quotation{qu'une source lointaine ne saurait
eteindre un incendie tout proche}. Même si des renforts arrivent, ils ne
vous seront d'aucun secours. Phuong Chinh, Ma Ky, pleins de cruauté,
avaient plongé notre peuple dans le malheur, s'étaient attiré la haine
de tous, avaient déterré les tombeaux de nos ancetres, capturé nos
femmes et nos enfant. Les vivants en pâtissaient, les morts ne pouvaient
digérer leur haine. Voyez les faits, examinez la conjoncture. Vous
verrez que vous devrez chatier Phuong Chinh, Ma Ky, les décapiter,
déposer leurs têtes à la porte de la citadelle, Pour éviter que toute la
garnison ne soit massacrée, pour pouvoir panser les blessures des vôtres

Nos deux pays renoueront leurs liens d'amitié, la guerre cessera pour
toujours. Si vous voulez retirer vos troupes, les routes comme les
jonques sont prêtes : voie terrestre ou maritime, à vous de choisir en
toute sécurité. Je me contenterai de mon rang de vassal, et paierai le
tribut d'usage. Si voulez faire autrement, disposez vos troupes en
position de bataille, que nos armées s'affrontent une bonne fois en
terrain découvert pour voir qui l'emportera, mais cessez de vous terrer
comme une vieille malade !

\stopsectionlevel

\startsectionlevel[title={LETTRE À VUONG
THONG},reference={lettre-à-vuong-thong-2}]

Les anciens disaient : \quotation{Ne talonnez pas un ennemi qui est à
bout}. J'aurais pu conduire mes trois à quatre cent mille hommes à
l'assaut de vos quatre citadelles et forteresses ; cependant, je sais
qu'un oiseau acculé à la mort peut encore user de son bec, qu'une bête
aux abois peut encore griffer, et je n'ai pas poussé mes troupes
victorieuses et mes volontaires de la mort à rechercher une victoire
futile. Néanmoins, une petite force, si résolue soit-elle, finit
toujours par être écrasée, On n'a jamais vu un œuf résister à un rocher

Pour le moment, ne parlons pas de l'assaut contre vos citadelles. Je
peux tout aussi bien ne plus me soucier de votre présence et faire
reposer nos armées, m'assurer le concours des hommes de vertu et de
talent, bien fourbir nos armes, entrainer nos hommes et nos éléphants,
apprendre aux officiers et soldats l'art militaire, cultiver en tous les
sentiments d'humanité et de justice, inculquer en la loyauté et la
franchise, l'affection vis-à-vis des supérieurs et le dévouement jusqu'à
la mort à leur chef. Telles seront mes armes pour affronter les ennemis.
Ceux qui seront contre moi périront, ceux qui seront avec moi vivront.
Telles seront mes atouts.

Si jamais votre gouvernement, débarrassé de ses soucis, nourrissant de
nouvelles ambitions, envoyait encore quelques dizaines de milliers
d'hommes nous envahir, ce serait pour nous un jeu que de nous y opposer.
Quant à vous, il n'est pas même pas besoin d'engager le combat pour vous
capturer. Entre ces deux façons d'agir, je n'ai pas encore choisi.

Je ne sais si vous-même appréciez bien le fait que nous nous
désintéressions de votre présence. Veuillez me donner votre avis, j'en
serais très honoré.

\stopsectionlevel

\startsectionlevel[title={LETTRE AUX OFFICIERS ET SOLDATS DES CITADELLES
DE THANH HOA-NGHÊ
AN},reference={lettre-aux-officiers-et-soldats-des-citadelles-de-thanh-hoa-nghê-an}]

Se sacrifier pour la patrie est le devoir sacré de tous les sujets,
juger des services rendus, les récompenser est pour l'État une tâche
quotidienne. Vous avez fait preuve de fidélité, d'héroïsme, combattu les
ennemis du roi, accompli de nombreux exploits.

Sous le régime prospère de l'ancienne dynastie\footnote{Actuellement :
  Hanoi.}, le Champa violant la loi du Ciel, avait envahi nos marches,
et vos ancetres, répondant à l'appel de la Patrie, avaient chassé les
agresseurs, préservé l'intégrité du pays ; la postérité et l'histoire
ont gravé à jamais le souvenir de leurs victoires. Aujourd'hui, les
envahisseurs Ming, au mépris de la volonté du Ciel, abusant de leur
puissance militaire pour prolonger la guerre, cherchent à assouvir leurs
ambitions territoriales et plongent notre peuple dans le malheur depuis
bientôt vingt ans. Mais le destin a changé, l'heur a succédé au malheur.
L'insurrection a éclaté et, comme un rouleau, a bousculé l'ennemi,
recouvrant en quelques mois toutes nos terres. Reste seule la citadelle
de Dong Quan, où le général ennemi Vuong Thong, rassemblant ses esprits
et son dernier souflle, continue à s'agiter avec, frénésie.

J'ai bien vu que les garnisons de la capitale, celles de Thien Truong,
Thien Vuong, comme tous ceux qui sont à la tête des grands offices, et
les membres de l'ancienne famille royale n'ont point encore déployé tous
les efforts pour arracher de belles victoires ; mais vous, sujets des
provinces éloignées, avez su continuer l'œuvre de vos ancetres ; unis,
fidéles à la patrie, pleins du désir de la venger, combatifs, vous avez
vaincu partout où vous avez com attu, et votre haute fidélité mérite
tous les éloges. J'ai \letterbar{} donc donné l'ordre de récompenser
dignement. Allez toujours de l'avant.

\stopsectionlevel

\startsectionlevel[title={LETTRE A VUONG
THONG},reference={lettre-a-vuong-thong}]

Une seule colonne ne saurait supporter tout un édifice en voie de
s'ecrouler, une pelletée de terre ne saurait maintenir une digue prête à
se rompre. Celui qui agit au-delà de ses forces court inévitablement à
l'échec.

Ne parlons plus du passé. Aujourd'hui, les choses en sont là. Votre seul
espoir réside dans la venue des renforts. Au 1er mois de cette année,
votre Empereur à mandé au duc de An Vien, au comte de Bao Dinh, au
maréchal Thoi, au mministre Hoang, au censeur Ly et au chef local Nguyen
Huan de rassembler leurs troupes et, au 4e mois, de les faire envahir
notre pays.

Au bout d'un mois, leurs hommes abordaient nos premiers postes. Nos
garnisons frontières les ont attirés jusqu'à la passe de Chi Lang. Au 2e
mois, en une seule bataille, nos troupes les ont mis en déroute,
décimant toute l'avant-garde, et le généralissime An Vien a été tué
sur-le-champ. Le 25, nous avons attaqué à nouveau, et toute l'armée
adverse a été dispersée, le comte de Bao Dinh fut tué et tous les
rescapés qui fuyaient dans les forêts ont été capturés.

Je n'ai point voulu que les choses en vinssent jusque-là. Ce sont
simplement nos officiers des frontières qui ont agi, et aggravé ainsi
mes fautes à votre égard. Vous, vous savez faire preuve d'humanité dans
votre commandement. En arrivant dans notre pays, Vous avez su vous
garder de vous fier seulement à l'action militaire. A lire votre
supplique à l'Empereur pour investir la dynastie des Tran, je vous sais
gré de vos bonnes intentions et ne saurais agir en ingrat.

Aujourd'hui, ce serait facile pour moi de mobiliser les forces de toute
une nation pour prendre d'assaut la petite citadelle de Dong Quan, Si
j'agis autrement, c'est en reconnaissance de votre générosité passée et
par déférence pour un grand pays. Si vous remettez en ordre vos troupes,
ne faites plus usage des armes, débloquez la citadelle, en accord avec
les engagements passés, vous pouvez ramener au pays tous vos hommes en
toute sécurité. Vous en aurez avec ce défaut des Han et des Tang qui
nourrissent de grandes ambitions, se réjouissaient des exploits
militaires, et rénoué avec la doctrine des Thang et Vu, soucieux de
restaurer les pays ruinés et les dynasties déchues. Quoi de plus beaux ?
Si vous tergiversez, j'ai peur que nos soldats et officiers, excedés par
des operations incessantes, impatients de retourner à leurs champs, ne
se décident à engager rapidement le combat. Je ne saurais alors les en
empecher. Ce serait trop tard. Et mes fautes à votre égard n'en seraient
qu'aggravées.

Veuillez bien me faire l'honneur d'une réponse.

\stopsectionlevel

\startsectionlevel[title={APPEL AUX HOMMES DE
TALENT},reference={appel-aux-hommes-de-talent}]

Toutes les citadelles sont tombées entre nos mains ans sauf Dong
Quan\footnote{Actuellement : Hanoi.}.

Je perds le sommeil et l'appétit, et me tracasse du matin au soir.
Surtout quand je ressens l'absence d'homme de grand talent. Je suis bien
le Chef, mais suis accablé de l'âge et peu doué, le savoir et
l'instruction me manquent, les tâches qui m'incombent sont trop
lourdes ; nous n'avons encore ni chancelier, ni ministres, ni maréchaux
et à peine un ou deux sur dix des postes dans notre administration ont
leurs responsables.

En toute modestie et sincérité, j'invite donc tous les héros et hommes
de talent, à associer leurs efforts aux nôtres afin de sauver le peuple,
qu'ils abandonnent leur retraite pour éviter au pays de s'enliser dans
le malheur. Mème si certains aspirent à faire Tu Hao ou Tu
Phong\footnote{Des ermites réputés.}, je les conjure de penser aux
malheurs du peuple, et quand notre œuvre commune sera accomplie, je
saurai respecter leur vocation et ne m'y opposerai pas, quand ils
voudront à nouveau répondre à l'appel des montagnes et des forêts.

\stopsectionlevel

\startsectionlevel[title={ÉDIT ADRESSÉ AUX HOMMES DE TALENT (Cu hièn
chiéu)},reference={édit-adressé-aux-hommes-de-talent-cu-hièn-chiéu}]

J'ai toujours pensé que la prospérité d'un règne dépend du talent des
hommes qui le servent, et que la promotion de tels hommes est capitale.

Tout souverain doit en faire sa préoccupation primordiale. Jadis, sous
les règnes les plus brillants, les hommes de talent se pressaient à la
Cour, les uns cédaient volontiers leur charge aux autres, de sorte qu'on
n'oubliait aucun homme capable, ne délaissait aucune tâche, et le pays
vivait dans la prospérité. Sous les Han et les Tang, tous cherchaient à
promouvoir des hommes capables, les uns aidaient les autres à l'élever à
un rang supérieur ; Tieu Ha avait présenté Tao Tham --- Nguy Tri, Tran
Binh --- Dich Nhan Kiet, Truong Cuu Linh --- Tieu Tung, Han Huu, des
hommes cetes de niveaux différents quant aux vertus et au talent, mais
tous capables. Aujourd'hui, assumant mes lourdes taches souveraines, je
tremble matin et soir. . Souffre, car j'en ai Soir comme si j'étais au
bord d'un gouffre, car j'en ai appelé en vain à la collaboration des
hommes de vertus et de talent. J'ordonne donc à tous les dignitaires
mandarins à partir du 3e grade l, militaires et civils, de présenter
chacun un homme, que ce soit à la Cour ou dans les villages, qu'il soit
déjà pourvu d'une charge publique ou non, pourvu qu'il ait des capacités
civiles ou militaires, je confierai à chacun une charge à la mesure de
son talent. D'autre part, selon la pratique ancienne, celui qui présente
un homme de vertu sera récompensé par le souverain. Si l'homme présenté
est de très grand talent et de très haute vertu, une forte récompense
sera accordée.

Néanmoins, les hommes de talent sont en nombre et il n'y a pas qu'une
seule façon de les promouvoir. Beaucoup d'hommes de talent occupent des
postes subalternes, beaucoup de héros vivent cachés dans les hameaux
retirés ou se trouvent dans les rangs des soldats, ou se sont retirés
dans les villages ; personne ne les a présentés, et si eux-mêmes ne se
mettent pas en avant, comment moi, souverain, puis-je les connaître{\bf
?}\footnote{typo : question mark missing} Dorénavant, ceux parmi les
hommes d'élite qui désirent devenir mes compagnons peuvent se présenter
d'eux-mêmes. Jadis Mao Toai\footnote{Antiquité chinoise.} sans se
formaliser avait bien de son propre chef demandé à suivre Binh Nguyen
Quan\footnote{Antiquité chinoise.}, Ninh Thich avait frappé sur la corne
du buffle en chantant pour attirer l'attention du duc Hoan des Te. Ces
personnages s'étaient-ils formalisés pour des choses futiles ? Cet édit
rendu public, que chaque mandarin à tous les échelons s'évertue à
présenter des hommes de talent et de vertu. Quant aux lettrés retirés au
fond des villages, qu'ils ne s'effarouchent pas de devoir
\quotation{vendre leur perle à la criée} pour éviter à moi, souverain,
la tristesse de penser que le pays manque d'hommes.

\stopsectionlevel

\startsectionlevel[title={ÉDIT SUR LA
MONNAIE},reference={édit-sur-la-monnaie}]

La monnaie, véritable sang qui féconde la vie du peuple ne saurait
manquer.

Notre pays produisait du cuivre en abondance, mais les Ho ayant détruit
toutes les pièces en cuivre, il en reste à peine un centième en
circulation et nous en manquons pour régler nos affaires civiles et
militaires. Serait-il donc si difficile d'avoir une monnaie qui circule
bien à la satisfaction du peuple ?

Récemment, quelqu'un m'a adressé une supplique proposant de remplacer
les pièces en métal par de la monnaie-papier. Jour et nuit, je
réfléchis, ne sachant à quoi me résoudre. Car l'usage du papier sans
valeur dans des transactions portant sur des choses qui en ont, risque
de faire perdre confiance au peuple. Cependant, jadis certains pensaient
déjà que l'or, l'argent, la soie, les peaux, la monnaie métallique, la
monnaie-papier se valaient, que faut-il en penser ?

J'ordonne donc à tous les dignitaires de la Cour, à tous les lettrés et
notables du pays, à tous ceux qui comprennent bien la situatio actuelle,
de discuter de ce problème pour éviter d'imposer la volonté d'un seul à
des millions de non-consentants et d'établir un système qui ait
l'assentiment du peuple et qui durera tout un règne.

Discutez-en le plus tôt possible, transmettez-moi le rapport de vos
discussions, je choisirai parmi les suggestions et déciderai moi-même.

\stopsectionlevel

\startsectionlevel[title={CONSEILS AU PRINCE
HÉRITIER\footnote{Rédigés au nom du Roi.}},reference={conseils-au-prince-héritier18}]

J'ai émergé des ronces, chassé les barbares, porté la cotte de mailles,
couché dans l'herbe, bravé les dangers, affronté lances et épées pour
balayer toutes les difficultés et créer ce royaume. Bâtir un empire
n'est guère facile ! Toi qui prends ma succession, tu dois graver dans
ta mémoire tous ces exploits, déploie tous les eflorts à respecter les
principes et méthodes qui régentent l'art militaire comme
l'administration, les choses publiques comme la conduite privée, évite
surtout de te plonger dans les plaisirs.

Cultive un tel esprit de fraternité pour entretenir la concorde parmi
tes proches ; que l'amour du peuple t'inspiré de nombreux actes de
générosité. Qu'aucune récompense ne soit accordée pour ton bon plaisir,
qu'aucun châtiment ne soit infligé pour une rancune personnelle. Ne
cours pas après l'argent et les biens précieux pour finir dans la
prodigalitéb évite la musique et les femmess pour ne pas sombrer dans La
luxure. Que ce soit pour promouvoir un homme de talent, prêter l'oreille
à un conseil, édicter une politique, un ordre, émettre une parole,
commetre un seul acte, garde le juste milieu, en conformité avec les
principes universelwment admis, de façon à satisfaire le Ciel et
répondre aux aspirations du peuple.

À ce prix seulement, le royaume pourra durer. Si, te fiant à ton seul
jugement, tu t'entoures de favoris, élimines mes compagnons, abandonnes
ma politique, enfreins mes principes familiaux, évites les proches gens
droits et intègres, pour te rapprocher des flagorneurs, n'agis que pour
ton plaisir, ne cours qu'après les spectacles, gaspilles le trésor
public, oublies toutes les difficultés du passé, il en sera comme les
anciens l'ont prédit : \quotation{Le père à tout préparé pour batir la
maison mais le fils n'a voulu ni s'occuper des fondations, ni ériger des
colonnes. Le père s'est appliqué à labourer, mais le fils n'a voulu ni
semer ni moissonner}. Comment pourras-tu exécuter mes volontés,
continuer mon œuvre, et perpétuer la dynastie ?

Car le peuple aime ceux qu'animent les vertus d'humanité, et, comme
l'océan qui porte la barque et peut la faire chavirer le peuple peut
porter le trône ou le faire sombrer. Le Ciel aide les gens de vertu,
mais la volonté du Ciel n'est pas toujours prévisible ; elle peut
changer à chaque moment si l'on ne mérite pas cette aide. Les anciens
rois Thuan, Vo, Thang, Van, qui étaient des saints n'en tremblaient pas
moins jour et nuit de manquer à leur devoir, ils s'appliquaient au
travail restaient parcimonieux dans leurs dépenses, s'évertuaient à
Vénérer le Ciel et à s'occuper du peuple, n'osaient rien négliger.

Que dire de ceux qui leur sont à tous points de vue inférieurs ? Souvent
les héritiers qui ont grandi dans la richesse ne savent forger leur
volonté, et si on oublie de les éduquer, de les prévenir dès le début,
pour leur inspirer une crainte salutaire du mal, de l'ardeur au travail,
comment pourraient-ils continuer l'œuvre de leurs prédécesseurs, faire
du bien au peuple ? On ne saurait éviter de leur donner des conseils.
Parce qu'il faut respecter les paroles de son père, le roi
Khai\footnote{Fils de l'Empereur Dai Vu de la dynastie des Ho.} put
régner longtemps ; parce qu'il sut continuer l'œuvre de ses
prédécesseurs, le roi Vu\footnote{Vu Vuong, fils de l'Empereur Van Vuong
  de la dynastie chinoise des Chu.} est resté célèbre.

Mon fils, garde bien en mémoire mes recommandations, suis l'exemple de
ceux qui t'ont précédé.

Tels sont mes conseils,

Traduction Nguyén Tchai Vien.

\thinrule

\stopsectionlevel

\startsectionlevel[title={GEOGRAPHIE DU
VIETNAM},reference={geographie-du-vietnam}]

\startsectionlevel[title={SUR LA CAPITALE THANG
LONG},reference={sur-la-capitale-thang-long}]

Ici, la terre cultivable est de couleur jaune, les riziéres
appartiennent à la catégorie \quotation{moyenne-supérieure}\footnote{Selon
  leur fertilité, les terres étaient divisées en 3 catégories
  (supérieures, moyennes, inférieures), chaque catégorie étant
  subdivisée en 3 échelons.}. Le quartier Tang Kiem fabrique des chaises
à porteurs, des armes et cuirasses, des objets de culte, des palanquins,
des parasols et autres articles d'apparat, le quartier de Yen Thai, du
papier. On tisse des étoffes fines et de la soie dans ceux de Thuy
Chuong et de Nghi Tam, celui de Ha Tan fabrique de la chaux tandis que
la rue de la Soie s'occupe des teintures. Des éventails sont fabriqués
dans le quartier de Ta Nhat. Le Lac de l'Ouest est riche en gros
poissons. On cultive des longaniers au quartier de Thinh Quang. Celui de
Duong Nhan vend des tuniques à la chinoise. C'est dans la capitale que
sont fabriqués les objets offerts en tribut à la Cour de Chine :
brocarts, broderies, encens et musc, divers objets d'or, d'argent et de
bronze.

Les commerçants des Deux Quang\footnote{Provinces chinoises de Kouang
  Toung et de Kouang Si.} habitent le quartier de Duong Nhan.

\stopsectionlevel

\stopsectionlevel

\stopsectionlevel

\stoptext
