\mainlanguage[fr]
% Enable hyperlinks
\setupinteraction
  [state=start,
  title={Féodédal},
  author={Philippe Fassier},
  keyword={jeux; labyrinthe; labyrôle; j&s},
  style=,
  color=,
  contrastcolor=]

\setuppagenumbering[location={footer,middle}]
%~ \setupbackend[export=yes]
\setupstructure[state=start,method=auto]

\setupwhitespace[medium]



\setuphead[chapter, section, subsection, subsubsection, subsubsubsection, subsubsubsubsection][number=no]



% \setupexternalfigures[directory={images, /home/user/images}]
\setupexternalfigures[directory={feodedal}]

\environment style

\starttext
\title[Féodédal]{Féodédal}

\placeaside[right,high]{}{
  \startaside[width=.3\textwidth]
    \unhyphenated{
      \startitemize[none, fit][stopper={}]
        \item  Philippe Fassier
        \item août-septembre 1984
        \item jeux & stratégie №28
      \stopitemize }
  \stopaside }
\startabstract
Le diabolique Philippe Fassier a encore frappé ! Vous vous retrouverez
d'ici peu perdu dans le inquiétantes ruines des pages suivantes. En
sortir ne sera pas si facile. Mais quant à y parvenir non seulement en
bonne santé mais de plus riche \ldots{} ce sera une autre histoire !

\stopabstract
\blank[big]

Vous voici à l'entrée d'un inquiétant château en ruines, détruit par
quelque séisme ou colère divine. Votre intrusion risque de réveiller des
légendes depuis longtemps oubliées\ldots{} Vous vous êtes aventuré dans
l'enceinte sans précaution ni matériel (cordes, lampes\ldots{}) et le
pont levis s'est effondré après votre passage. C'était le seul moyen
d'accès.

{\bf but du jeu:} ressortir du château sain et sauf et peut être chargé
de trésors.

{\bf déroulement de la partie:} vous débutez en 1. À chaque numéro
rencontré lors de votre périple, vous vous reportez à la liste de
description ci-dessous pour connaître l'événement correspondant (vous
trouverez la signification des symboles utilisés dans le tableau).
Lorsque, dans la liste des événements, vous rencontrez un nombre de 100
à 200, vous vous reportez aux notes de résolution, page 110, pour
connaître le résultat de votre action.

Sauf indication contraire, vous ne pouvez pas franchir les zones noires
ni sauter les murs.

N'oubliez pas que votre but premier est de ressortir du château
vivant\ldots{} et en bonne santé.

\startitemize[packed]
\item
  Source: Scans de
  \useURL[url1][https://abandonware-magazines.org]\from[url1]
\item
  Font:
  \useURL[url2][https://speakthesky.itch.io/typeface-dicier][][Dicier,
  by Speak the Sky]\from[url2], licensed under
  \useURL[url3][https://creativecommons.org/licenses/by/4.0/][][CC BY
  4.0]\from[url3]
\stopitemize

\page

{\externalfigure[01.jpg][width=1.15\textwidth][centered]}

{\externalfigure[02.jpg][width=1.15\textwidth][centered]}

\page

\subsubsection[title={Symboles utilisés},reference={symboles-utilisés}]

\startplacetable[location=none]
\startxtable
\startxtablebody[body]
\startxrow
\startxcell[align=right] → n \stopxcell
\startxcell[align=right] Se rendre à la note n. \stopxcell
\startxcell[align=right] ← n \stopxcell
\startxcell[align=right] Lire la note n. \stopxcell
\stopxrow
\startxrow
\startxcell[align=right] si ← n \stopxcell
\startxcell[align=right] Si vous avez lu la note n \ldots{} \stopxcell
\startxcell[align=right] \dice{HEARTS} \stopxcell
\startxcell[align=right] Vous trouvez \ldots{} \stopxcell
\stopxrow
\startxrow
\startxcell[align=right] \dice{KNIVES} \stopxcell
\startxcell[align=right] Vous êtes mort. \stopxcell
\startxcell[align=right] \dice{0} \stopxcell
\startxcell[align=right] Vous pouvez \ldots{} \stopxcell
\stopxrow
\startxrow
\startxcell[align=right] \dice{ANY} \stopxcell
\startxcell[align=right] Entrer. \stopxcell
\startxcell[align=right] \dice{COIN} \stopxcell
\startxcell[align=right] Posséder. \stopxcell
\stopxrow
\startxrow
\startxcell[align=right] ↑ \stopxcell
\startxcell[align=right] Se diriger vers\ldots{} \stopxcell
\startxcell[align=right] R \stopxcell
\startxcell[align=right] Revenir sur ses pas\ldots{} \stopxcell
\stopxrow
\startxrow
\startxcell[align=right] \dice{CLUBS} \stopxcell
\startxcell[align=right] C'est un cul de sac. \stopxcell
\startxcell[align=right] X \stopxcell
\startxcell[align=right] Impossible. \stopxcell
\stopxrow
\startxrow
\startxcell[align=right] \dice{DIAMONDS} \stopxcell
\startxcell[align=right] Passage souterrain. \stopxcell
\startxcell[align=right] P \stopxcell
\startxcell[align=right] Une pièce inoccupée. \stopxcell
\stopxrow
\stopxtablebody
\startxtablefoot[foot]
\startxrow
\startxcell[align=right] \dice{CASTLES} \stopxcell
\startxcell[align=right] Il ne se passe rien. \stopxcell
\startxcell[align=right]  \stopxcell
\startxcell[align=right]  \stopxcell
\stopxrow
\stopxtablefoot
\stopxtable
\stopplacetable

\subsection[title={Descriptions},reference={descriptions}]

\startitemize[n][stopper=.,width=2.0em]
\item
  Vous êtes \dice{ANY} dans les ruines, le pont-levis s'écroule, XR.
\item
  Un puit, la chaîne est assez longue pour ↑ 12.
\item
  P sombre → 100.
\item
  Si vous venez de 22 → 126; sinon → 194.
\item
  P, une porte sur le côté, si vous l'ouvrez → 164.
\item
  Vous tentez de ↑ 14 → 184; vers 11 → 200; sinon \dice{CASTLES}.
\item
  ← 147, X de bouger, ↑.
\item
  P, un passage ↑ 13.
\item
  \dice{HEARTS} un \dice{DIAMONDS}, si vous \dice{ANY} → 134.
\item
  X ↑ 5, mais \dice{0} ↑ 18.
\item
  Une faille a fait s'écrouler l'escalier, mais le passage est
  possible;\crlf
  si vous tentez de ↑ 6 → 200; si vous R → 118.
\item
  Une résurgence rejoint le torrent → 104.
\item
  L'escalier ↑ 14.
\item
  Si vous tentez de ↑ 6 → 113; sinon \dice{0} R.
\item
  P, → 101.
\item
  Un escalier très sombre \ldots{} → 176.
\item
  P, un socle vide → 127.
\item
  Un toboggan X à remonter, \dice{0} R ou descendre.
\item
  Le mur en ruine permet de ↑ 36, si vous tentez → 116;\crlf
  sinon un escalier et un \dice{DIAMONDS} ↑ 23.
\item
  Rien de spécial.
\item
  Ce couloir ↑ 22.
\item
  Jolie vue sur la chute \ldots{}
\item
  X ↑ autre part que 19 ou 16.
\item
  Une porte X à ouvrir.
\item
  ← 120, si vous ↑ le bras de DRYH → 173;\crlf
  si vous \dice{ANY} dans la bouche → 121.
\item
  \dice{HEARTS} l'aire d'un aigle → 107.
\item
  La carrière de rubis cubiques, X de détacher la moindre gemme !\crlf
  → 106.
\item
  Un \dice{DIAMONDS}, si vous \dice{ANY} → 136.
\item
  ← 120; → 148.
\item
  \dice{HEARTS} un \dice{DIAMONDS}, si vous \dice{ANY} → 165.
\item
  Si vous avez l'œil de DRYH → 183; sinon → 153.
\item
  Les deux portes ne peuvent s'ouvrir que du côté apparent.
\item
  Si vous \dice{ANY} dans le \dice{DIAMONDS} → 154; sinon → 101.
\item
  Ce couloir ↑ 33.
\item
  Une eau noire vous sépare de 44; si vous R → 118;\crlf
  si vous ↑ 44 en barque → 146.
\item
  \dice{HEARTS} l'aire d'un aigle - 117.
\item
  Si vous \dice{ANY} dans le \dice{DIAMONDS} → 182.
\item
  L'eau noire ne semble pas profonde pour ↑ 48; si vous traversez → 145.
\item
  Une porte ↑ 55; la porte se referme, XR.
\item
  → 102.
\item
  Vous avez la corde \ldots{} alors \dice{0} ↑ 40 ou 42.
\item
  → 102.
\item
  Si vous \dice{ANY} → 135.
\item
  X ↑ 35, l'eau est trop profonde.
\item
  La sortie d'un \dice{DIAMONDS} sous une cascade; le \dice{DIAMONDS} ↑
  16.
\item
  Rien de spécial ici → 108.
\item
  → 137.
\item
  \dice{CASTLES}.
\item
  → 190.
\item
  Dans la grotte, deux yeux vous observent, vous R → 112;\crlf
  vous ↑ les yeux → 195.
\item
  Un passage étroit ↑ 53.
\item
  Une tour avec un escalier.
\item
  Un navire échoué sur la falaise → 189.
\item
  Un arbre se retient pour ne pas tomber avec la falaise → 158.
\item
  Un libre → 159.
\item
  Si vous sautez pour vous ↑ 57 → 172; sinon \dice{CLUBS}.
\item
  P, \dice{CASTLES}; X ↑ 56; X ↑ 47.
\item
  Une porte fermée, si vous l'ouvrez → 163.
\item
  Un cabanon sur lequel est inscrit « CAISSE » → 167.
\item
  Un petit volcan en activité → 111.
\item
  Sur l'écriteau est inscrit « SORTIE » → 149.
\item
  L'entrée d'une église en ruines → 133.
\item
  P, → 171.
\item
  Un \dice{DIAMONDS} → 155.
\item
  Vous êtes dans la nef → 157.
\item
  Une fosse à serpents → 191.
\item
  Vous êtes dans la nef → 109.
\item
  Un \dice{DIAMONDS} ↑ 70.
\item
  Une machine, si vous utilisez la corde pour ↑ 71 → 185; sinon
  \dice{CLUBS}.
\item
  ← 147; toutefois, un escalier et un \dice{DIAMONDS} ↑ 68.
\item
  Rien de spécial, X ↑ 69.
\item
  Si ← 124, \dice{0} passer, sinon \dice{CLUBS}, de; la lave est en
  fusion.
\item
  X de descendre.
\item
  Si vous tentez de ↑ 73 → 186, sinon \dice{CLUBS}.
\item
  Des éboulis bouchent un \dice{DIAMONDS} \dice{CLUBS}.
\item
  La demeure d'un vieux sorcier; si vous \dice{ANY} → 188; sinon
  \dice{CLUBS}.
\stopitemize

\page

\subsection[title={page 110},reference={page-110}]

\startitemize[n][start=100,stopper=.,width=2.5em]
\item
  Dans l'obscurité, deux portes;\crlf
  si ↑ la droite → 151; si ↑ la gauche → 196.
\item
  \dice{CASTLES}.
\item
  Une corde peut vous mener à 41.
\item
  Le socle semble fait pour lui, vous le posez? oui → 169; non → 118.
\item
  La chaîne tombe avec fracas dans l'eau, XR.
\item
  L'estrade sur laquelle vous êtes, vous emmène en 20.
\item
  \dice{0} ↑ 20 ou 29 ou autre en R.
\item
  et l'aigle est là \ldots{} ← 141 il vous emmène en 46.
\item
  X d'ouvrir les deux portes \ldots{} → 4.
\item
  \dice{CASTLES} \dice{CLUBS}; il y a un vase avec deux ailes, rempli de
  vif-argent, il est insoulevable, si vous R → 118, si vous videz le
  vase → 180.
\item
  Mais l'araignée, elle, ne vous épargne pas ! \dice{KNIVES}.
\item
  ← 141; il redouble de colère et rend X le passage en 72 → 144.
\item
  → 118.
\item
  → 184.
\item
  Si vous mettez l'œil de DRYH dans le trou → 142; sinon → 101.
\item
  Il vous soulève ! Vous êtes à quatre mètres du sol; vous lâchez ?\crlf
  oui → 174; non → 197.
\item
  Vous arrivez en haut de la tour sans dommage.
\item
  \dice{HEARTS} un œuf énorme, vous le prenez → 177; vous R → 112.
\item
  \dice{CASTLES}, \dice{0} R.
\item
  X de passer, \dice{0} que sauter dans le vide \dice{KNIVES}.
\item
  DRYH, père du royaume, pétrifié lors d'un combat de géants.
\item
  Un \dice{DIAMONDS} sombre et raide, vous tombez → 178.
\item
  Le \dice{DIAMONDS} ↑ 28.
\item
  Le \dice{DIAMONDS} ↑ 45.
\item
  Le volcan s'apaise, après un refroidissement, \dice{0} passer.
\item
  Si vous \dice{COIN} l'œuf d'aigle → 175; sinon 119.
\item
  Si vous ne \dice{COIN} pas l'œil de DRYH, X de passer entre les
  statues → 187.
\item
  Si vous \dice{COIN} l'œuf d'aigle → 103; sinon \dice{CASTLES}
  \dice{CLUBS}.
\item
  ← 141, ← 152, en échange il désire l'œuf d'aigle, si vous refusez
  \dice{0} R, sinon → 168.
\item
  D'un geste, il vous tue \dice{KNIVES}.
\item
  X de prévoir, vous tombez dans le vide \dice{KNIVES}.
\item
  Si vous \dice{ANY} dans la nef → 166; sinon → 101.
\item
  Un aigle fond sur vous, il vous emmène en 46 avec l'œuf.
\item
  Si vous \dice{ANY} dans la nef → 101; sinon \dice{0} R.
\item
  Le \dice{DIAMONDS} ↑ 43.
\item
  Le \dice{DIAMONDS} ↑ 9.
\item
  Le \dice{DIAMONDS} ↑ la bouche en 25.
\item
  Vous êtes mouillé, vous glissez et tombez en 57;\crlf
  si vous \dice{COIN} l'œuf, il se brise.
\item
  \dice{0} ↑ 11, mais attention au vertige.
\item
  Le vase s'envole vers le ciel à une vitesse vertigineuse,
  \dice{CASTLES} \dice{0} R.
\item
  Une eau noire déferle sur vous, \dice{KNIVES}.
\item
  Apparemment vous le dérangez dans son travail \ldots{}.
\item
  X de le retirer ! Le sol tremble, \dice{HEARTS} un \dice{DIAMONDS} à
  côté → 179.
\item
  Il vous expédie en 52.
\item
  Si vous ne \dice{COIN} pas l'œuf d'aigle, X de partir \dice{KNIVES};
  sinon → 192.
\item
  \dice{CASTLES}, mais vous êtes trempé jusqu'aux cuisses !
\item
  \dice{CASTLES}, mais arrivé en 44, la barque ↑ 35, XR.
\item
  Comment diable êtes- vous arrivé-là ?!
\item
  \dice{HEARTS} une gemme de diamant dans l'œil du Père DRYH.
\item
  Si ← 145, une semaine après vous perdez l'usage de vos jambes;\crlf
  si ← 192, vous devenez aveugle;\crlf
  si ← 173, votre bras droit s'atrophie;\crlf
  si ← 121, vous devenez muet;\crlf
  si ← 176, vous devenez claustrophobe;\crlf
  si ← 138, vous avez la phobie des hauteurs;\crlf
  sinon, tout va bien!
\item
  \dice{CASTLES}, \dice{0} R, l'œuf est intact \ldots{}.
\item
  P, un trou dans le mur → 138.
\item
  Il vous demande : « Que voulez-vous? »
\item
  \dice{CASTLES}, \dice{0} ↑ 32 (haut).
\item
  Le \dice{DIAMONDS} ↑ 37.
\item
  Dans le noir, une grille se referme derrière vous, XR → 193.
\item
  L'escalier tremble mais \dice{0} passer.
\item
  \dice{CASTLES}, il n'y a rien, pourtant \dice{HEARTS} un coffret
  rempli de joyaux!
\item
  \dice{0} ↑ 52, le \dice{DIAMONDS} est fermé, \dice{CLUBS}.
\item
  Tout ce que \dice{0} déchiffrer :

  \startblockquote
  « Celui qui désire la richesse embarquera; s'il convoite la gloire, il
  attendra le neuvième mois et s'il veut la liberté, il volera les ailes
  de mercure ».
  \stopblockquote
\item
  \dice{HEARTS} un trou sur le front, si vous \dice{COIN} l'œil de DRYH
  → 114, sinon → 101.
\item
  ← 141, ← 152, → 168.
\item
  Une fois, ça va, deux fois. l'escalier s'écroule \dice{KNIVES}.
\item
  Elle résiste, si vous insistez → 140.
\item
  P sombre, un rai de lumière passe sous une porte; vous R → 118; vous
  \dice{ANY} → 130.
\item
  Un \dice{DIAMONDS} sombre et raide, vous ↑ un carrefour dans le
  noir;\crlf
  vous ↑ la droite → 122; vous ↑ la gauche → 136.
\item
  Il sombra, mais ne tombera point.
\item
  Si vous ne \dice{COIN} aucune gemme, le guide ouvre une trappe sur un
  \dice{DIAMONDS} et vous y tombez → 193; sinon → 199.
\item
  Vous répondez :\crlf
  « la richesse » → 143\crlf
  « la gloire » → 181\crlf
  « la liberté » → 170.
\item
  L'œuf se dématérialise! → 105.
\item
  Il éclate de rire puis → 198.
\item
  \dice{CASTLES}, rien que des graffitis cabalistiques \ldots{}
  \dice{CLUBS}.
\item
  OK, \dice{CASTLES}, mais XR.
\item
  \dice{0} ↑ 32 (bas), \dice{CASTLES}.
\item
  Vous vous retrouvez en 70, il ↑ le ciel. si vous \dice{COIN} l'œuf, il
  s'est écrasé.
\item
  Avec ça, vous survivrez une semaine, car → 119.
\item
  L'escalier ↑ un \dice{DIAMONDS}; après une très longue marche dans le
  noir → 123.
\item
  OK; vous R en 19 ? oui → 150; vous attendez → 132.
\item
  Si vous \dice{COIN} l'œuf, il se brise; si vous \dice{COIN} l'œil de
  DRYH, vous le perdez; → 122.
\item
  Il ↑ 54.
\item
  Vidé du lourd liquide, il semble attiré vers le ciel; vous lâchez →
  139; vous ne lâchez pas → 115.
\item
  Il réfléchit et répond : « Vous la trouverez seul! » → 129.
\item
  Le \dice{DIAMONDS} ↑ 33.
\item
  La main de pierre s'abat sur vous, \dice{KNIVES}.
\item
  Les ruines s'écroulent, la toile d'araignée vous sauve de la chute
  mortelle → 110.
\item
  La machine ne bronche pas, mais arrivé en 71, elle rembobine la corde,
  XR.
\item
  Si vous \dice{COIN} l'œuf, il se brise; \dice{0} ↑ 73 mais XR.
\item
  \dice{0} ↑ 30 ou 22.
\item
  Si vous \dice{COIN} l'œuf d'aigle → 128; sinon → 161.
\item
  Le passage pour ↑ 51 s'écroule, XR → 131.
\item
  Une porte X à ouvrir, \dice{CLUBS}.
\item
  X R, \dice{KNIVES}.
\item
  En le jetant au feu, une vive explosion \ldots{} → 124.
\item
  Le \dice{DIAMONDS} ↑ 66.
\item
  Si vous \dice{COIN} l'œil de DRYH, \dice{0} passez entre les statues;
  sinon → 125.
\item
  \dice{CASTLES}, c'est une statue, les yeux sont en diamant, X à
  retirer → 160.
\item
  Un escalier dans le noir ↑ 5.
\item
  Trop tard, si vous lâchez, \dice{KNIVES};\crlf
  sinon \dice{KNIVES} à partir de 10 000 mètres d'altitude ! \ldots{}
\item
  Il vous expédie en 7.
\item
  Vous lui donnez l'œil de DRYH, (ou autre si vous \dice{COIN}), puis
  \dice{0} passer.
\item
  Si ← 156 → 162; sinon → 156.
\stopitemize

\stoptext
