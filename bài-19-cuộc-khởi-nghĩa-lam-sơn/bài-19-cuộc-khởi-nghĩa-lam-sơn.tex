% Enable hyperlinks
\setupinteraction
  [state=start,
  title={Bài 19 CUỘC KHỞI NGHĨA LAM SƠN (1418-1427)

},
  author={foo},
  style=,
  color=,
  contrastcolor=]

\setuppagenumbering[location={footer,middle}]
%~ \setupbackend[export=yes]
\setupstructure[state=start,method=auto]

\setupwhitespace[medium]



\setuphead[chapter, section, subsection, subsubsection, subsubsubsection, subsubsubsubsection][number=no]



% \setupexternalfigures[directory={images, /home/user/images}]
%~ \setupexternalfigures[directory={[[foobar]]}]
\setupexternalfigures[directory={bài-19-cuộc-khởi-nghĩa-lam-sơn}]


\environment style
\environment colorbrouwer


\starttext
\title[Bài 19 CUỘC KHỞI NGHĨA LAM SƠN (1418-1427)

]{Bài 19 CUỘC KHỞI NGHĨA LAM SƠN (1418-1427)

}

\placeaside[right,high]{}{
  \startaside[width=.3\textwidth]
    \unhyphenated{
      \startitemize[none, fit][stopper={}]
        \item  foo
        \item bar
      \stopitemize }
  \stopaside }

\startsectionlevel[title={I. THÒI Kì Ỏ MỀN TÂY THANH HOÁ
(1418---1423)},reference={i.-thòi-kì-ỏ-mền-tây-thanh-hoá-14181423}]

\startsectionlevel[title={1. Lê Lợi dựng cờ khởi
nghĩa},reference={lê-lợi-dựng-cờ-khởi-nghĩa}]

Lê Lợi (1385---1433) là một hào trưởng có uy tín lớn ở vùng Lam Sơn
(Thanh Hoá). Trước cảnh nước mất, nhân dân lầm than, ông đã dôc hết tài
sản đê chiêu tập nghĩa sĩ, bí mật liên lạc với các hào kiệt, xây dựng
lực lượng và chọn Lam Sơn làm căn cứ cho cuộc khởi nghĩa. Lam Sơn nằm
bên tả ngạn sông Chu, nối liền giưa đồng bằng với miền núi và có địa thế
hiểm tr cũng là nơi giao tiếp của các dân tộc Việt, Mường, Thái.

\startblockquote
Ông thường nói với mọi người :\crlf
\quotation{Bậc trượng phu sinh ở đời phải cứu nạn lớn,\crlf
lập công to, để tiếng thơm hàng nghìn thuở,\crlf
chứ đâu lại xun xoe đi phục dịch kẻ khác}.

(Khâm định Việt sử thông giám cương mục)
\stopblockquote

Nghe tin Lê Lợi đang chuẩn bị khởi nghĩa ở Lam Sơn, nhiều người yêu nước
từ các địa phương đã tìm về hội tụ ngày càng đông, trong đó có Nguyễn
Trãi.

\startblockquote
Nguyễn Trãi (1380--1442) là con Nguyễn Phi Khanh, cả hai cha con đều đỗ
đại khoa và làm quan thời Hồ. Ông học rộng, tài cao, có lòng yêu nước
thương dân hết mực. Quân Minh tìm đủ mọi cách để dụ dỗ ông nhưng đều
thất bại. Từ thành Đông Quan, ông bí mật trốn vào Lam Sơn theo Lê Lợi
khởi nghĩa và dâng bản Bình Ngô sách (Kê'sách đánh quân Ngô).
\stopblockquote

Đầu năm 1416, Lê Lợi cùng 18 người trong bộ chỉ huy cuộc khởi nghĩa đã
tổ chức hội thề ở Lũng Nhai (Thanh Hoá) và đọc bài văn thề:

\startblockquote
\quotation{Tôi là phụ đạo Lê Lợi cùng Lê Lai\ldots{}, Nguyễn Trãi, Đinh
Liệt, Lưu Nhân Chú. 19 người tuy họ hàng quê quán khác nhau, nhưng kết
nghĩa thân nhau như một tổ liền cành. Phận vinh hiển có khác nhau mong
có tình như cùng chung một họ\ldots{} chung sức đồng lòng, giữ gìn đất
nước, làm cho xóm làng được ăn ở yên lành. Thề sôhg chết cùng nhau,
không dám quên lời thề son sắt\ldots{} Kính xỉn có lời thề}.

(Lam Sơn thực lục)
\stopblockquote

\startitemize[packed]
\item
  Vì sao hào kiệt khắp nơi tìm vê Lam Sơn ?
\stopitemize

Ngày 2 tháng 1 năm Mậu Tuất (7-2-1418), Lê Lợi dựng cờ khởi nghĩa ở Lam
Sơn và tự xưng là Bình Định Vương.

\stopsectionlevel

\startsectionlevel[title={2. Những nõm đầu hoạt động của nghĩa quân Lam
Sơn},reference={những-nõm-đầu-hoạt-động-của-nghĩa-quân-lam-sơn}]

Những ngày đầu khởi nghĩa, lực lượng còn yếu, nghĩa quân Lam Sơn đã gặp
rất nhiều khó khăn, nguy nan. Quân Minh nhiều lần tấn công, bao vây căn
cứ Lam Sơn. Nghĩa quân ba lần phải rút lên núi Chí Linh (Lang Chánh,
Thanh Hoá) và phải liên tiếp chống lại sự vây quét của quân giặc. Trong
gian khổ đa có rất nhiều tấm gương chiên đấu hi sinh dũng cảm, tiêu biểu
là Lê Lai.

Giữa năm 1418, quân Minh huy động một lực lượng bao vầy chặt căn cứ Chí
Linh, quyết bắt giết Lê Lợi. Trước tình hình nguy cấp đó, Lê Lai đã cải
trang làm Lê Lợi, chỉ huy một toán quân liều chết phá vòng vây quân
giặc. Lê Lai cùng toán quân cảm tử đã hi sinh. Quân Minh tưởng rằng đã
giết được Lê Lợi nên rút quân.

\startblockquote
Lê Lai người dân tộc Mường, quê ở Dựng Tú (Ngọc Lặc---Thanh Hoá).\crlf
Gia đình ông có 5 người tham gia nghĩa quân Lam Sơn thì 4 người đã hỉ
sinh trong chiến đấu.
\stopblockquote

Cuôì năm 1421, quân Minh huy động hơn 10 vạn lính mở cuộc vây quét lớn
vào căn cứ của nghĩa quân. Lê Lợi lại phải rút quàn lên núi Chí Linh.
Tại đây, nghĩa quân đã trải qua muôn vàn khó khăn, thiếu lương thực trầm
trọng, đói, rét. Lê Lợi phải cho giết cả voi, ngựa (kể cả con ngựa của
ông) để nuôi quân.

\startblockquote
\quotation{Khỉ Linh Sơn\footnote{Linh Sơn : núi Chí Linh, thuộc huyện
  Lang Chánh (Thanh Hoá).} lương hết mấỵ tuần,\crlf
Khi Khôi Huyện\footnote{Khôi Huyện : còn gọi là Khôi Sách, thuộc miền
  Tây của Ninh Bình.} quân không một đội}.

(Bình Ngô đại cáo)
\stopblockquote

Mùa hè năm 1423, Lê Lợi đề nghị tạm hoà và được quân Minh chấp thuận.
Tháng 5--1423, nghĩa quân trở về căn cứ Lam Sơn.

\startitemize[packed]
\item
  Tại sao Lê Lợi đề nghị tạm hoà với quân Minh ?
\stopitemize

Cuối năm 1424, do bị thát bại trong âm mưu mua chuộc Lê Lợi, quân Minh
trở mặt, tấn công nghĩa quân. Cuộc khởi nghĩa Lam Sơn chuyên sang giai
đoạn mới.

\stopsectionlevel

\startsectionlevel[title={CÂU HỎI},reference={câu-hỏi}]

\startitemize[n,packed][stopper=.]
\item
  Em hãy trình bày tóm tắt diễn biến cuộc khởi nghĩa Lam Sơn trong giai
  đoạn 1418-1423.
\item
  Ẹm có nhận xét gì về tinh thần chiến đấu của nghĩa quân Lam Sơn trong
  những năm 1418- 1423 ?
\item
  Tại sao lực lượng quân Minh rất mạnh nhưng không tiêu diệt được nghĩa
  quân mà phải chấp nhận để nghị tạm hoà của Lê Lợi ?
\stopitemize

\stopsectionlevel

\stopsectionlevel

\startsectionlevel[title={II. GIẢI PHÓNG NGHỆ AN, TÂN BÌNH, THUẬN HOÁ VÀ
TIỂN QUÂN PA BẮC
(1424---1426)},reference={ii.-giải-phóng-nghệ-an-tân-bình-thuận-hoá-và-tiển-quân-pa-bắc-14241426}]

\startsectionlevel[title={1. Giải phóng Nghệ An (năm
1424)},reference={giải-phóng-nghệ-an-năm-1424}]

Trước tình hình quân Minh tấn công nghĩa quân, Nguyễn Chích đề nghị tạm
rời núi rừ ng Thanh Hoá, chuyên quân vào Nghệ An là nơi đất rộng, người
đông và cũng rất hiểm yếu, để dựa vào đó quay ra đánh lấy Đông Đô.

\startblockquote
Nguyễn Chích là một nông dân nghèo ở Thanh Hoá, đã từ ng lãnh đạo cuộc
khởi nghĩa chôhg quân Minh ở nam Thanh Hoá và hoạt động ở vùng bắc Nghệ
An. Năm 1420, Nguyễn Chích đem quân gia nhập nghĩa quân Lam Sơn.
\stopblockquote

Kế hoạch chuyên địa bàn hoạt động của Nguyễn Chích được Lê Lợi chấp
nhận. Nghĩa quân theo đường núi tiến vào miền Tây Nghệ An. Ngày
12-10-1424, nghĩa quân bất ngờ tập kích đồn Đa Căng (Thọ Xuân---Thanh
Hoá) và thắng lợi giòn giã, sau đó hạ thành Trà Lân ở thượng lưu sông
Lam, buộc địch phải đầu hàng sau hai tháng vây hãm.

Trên đà thắng đó, nghĩa quân tiến đánh quân giặc ở Khả Lưu (tả ngạn sông
Lam thuộc Anh Sơn, Nghệ An). Bằng kế nghi binh, nghĩa quân đánh bại quân
Trần Trí ở Khả Lưu, Bồ Ải. Được nhân dân ủng hộ, chỉ trong một thời gian
ngắn, phần lớn Nghệ An được giải phóng, quân giặc phải rút vào thành cố
thủ. Lê Lợi siết chặt vòng vây thành Nghệ An, tiến đánh Diễn Châu rồi
thừ a thắng tiến quân ra Thanh Hoá. cả vùng Diễn Châu, Thanh Hoá được
giải phóng chỉ trong vòng không đầy một tháng.

\startitemize[packed]
\item
  Em có nhận xét gì về kế hoạch của Nguyễn Chích ?
\stopitemize

\stopsectionlevel

\startsectionlevel[title={2. Giải phóng Tân Bình, Thuận Hoá (năm
1425)},reference={giải-phóng-tân-bình-thuận-hoá-năm-1425}]

Tháng 8-1425, các tướng Trần Nguyên Hãn, Lê Ngân\ldots{} được lệnh chỉ
huy một lực lượng mạnh từ Nghệ An tiến vào Tân Bình (Quảng Bình, Bắc
Quảng Trị) và Thuận Hoá (Thừ a Thiên Huế). Nghĩa quân nhanh chóng đập
tan sức kháng cự của quân giặc, giải phóng Tân Bình, Thuận Hoá.

Như vậy, trong vòng 10 tháng (từ tháng 10-1424 đến tháng 8-1425), nghĩa
quân Lam Sơn đã giải phóng được một khu vực rộng lớn từ Thanh Hoá vào
đến đèo Hải Vân. Quân Minh chỉ còn giữ được mây thành luỹ bị cô lập và
bị nghĩa quân vây hãm.

\startitemize[packed]
\item
  Em hãy trình bày tóm tắt các chiến thắng của nghĩa quân Lam Sơn từ
  cuối năm 1424 đến cuối năm 1425.
\stopitemize

\stopsectionlevel

\startsectionlevel[title={3. Tiến quân ra Bắc, mở rộng phạm vi hoạt động
(cuối năm
1426)},reference={tiến-quân-ra-bắc-mở-rộng-phạm-vi-hoạt-động-cuối-năm-1426}]

Tháng 9--1426, Lê Lợi và bộ chỉ huy quyết định mở cuộc tiến quân ra Bắc.
Nghĩa quân chia làm ba đạo.

\startblockquote
Đạo thứ nhát tiến quân ra giải phóng miền Tây Bắc, ngăn chặn viện binh
giặc từ Vân Nam sang. Đạo thứ hai có nhiệm vụ giải phóng vùng hạ lưu
sông Nhị và chận đường rút quân của giặc từ Nghệ An về Đông Quan, ngăn
chặn viện binh từ Quảng Tây sang. Đạo thứ ba tiến thăng ra Đông Quan.
\stopblockquote

\startplacefigure[title={Hình 41---Lược đồ đường tiến quân ra Bắc của
nghĩa qụân Lam Sơn}]
{\externalfigure[maps/map1.png]}
\stopplacefigure

Nhiệm vụ của cả ba đạo quân là tiến sâu vào vùng chiếm đóng của địch,
cùng với nhân dân nổi dậy bao vây đồn địch, giải phóng đất đai, thành
lập chính quyền mới, chặn đường tiếp viện của quân Minh từ Trung Quốc
sang.

\startitemize[packed]
\item
  Dựa vào lược đổ, em hãy trinh bày kể hoạch tiến quân ra Bắc của Lê
  Lợi. Nhận xét về kể hoạch đó.
\stopitemize

Nghĩa quân tiến đến đâu cũng được nhân dân nhiệt tình ủng hộ về mọi mặt.

\startblockquote
Nhiều tấm gương yêu nước xuất hiện như bà Lương Thị Minh Nguyệt ở làng
ChuếCầu (Ý Yên---Nam Định) bán rượu, thịt ỏ thành Cổ Lông, lừ a cho giặc
ăn uôhg no say, rồi bí mật quẳng xuống kênh chảy ra sông Đáy ; hoậc cô
gái người làng Đào Đặng (Hưng Yên) xinh đẹp, hát hay thường được mời đêh
hất mua vui cho giặc. Đêm đến, sau những buổi ca hát, tiệc tùng, nhiều
kẻ chui vào bao vải ngủ để tránh muỗi. Cô cùng trai làng bí mật khiêng
quăng xuống sông.
\stopblockquote

Được sự ủng hộ tích cực của nhân dân, nghĩa quân chiến thắng nhiều trận
lớn, quân Minh lâm vào thế phòng ngự, rút vào thành Đông Quan cố thủ.
Cuộc kháng chiến chuyển sang giai đoạn phản công.

\stopsectionlevel

\startsectionlevel[title={CÂU HỎI},reference={câu-hỏi-1}]

\startitemize[n,packed][stopper=.]
\item
  Em hãy trình bày tóm tắt diễn biến cuộc khởi nghĩa Lam Sơn từ cuối năm
  1424 đến cuối năm 1426.
\item
  Em hãy nêu những dẫ n chứng vế sự ủng hộ của nhân dân trong cuộc khởi
  nghĩa Lam Sơn từ cuối năm 1424 đến cuối năm 1426.
\stopitemize

\stopsectionlevel

\stopsectionlevel

\startsectionlevel[title={III. KHỎI NGHĨA LAM SƠN TOÀN THẮNG (Cuối Năm
1426---cuối Năm
1427)},reference={iii.-khỏi-nghĩa-lam-sơn-toàn-thắng-cuối-năm-1426cuối-năm-1427}]

\startsectionlevel[title={1. Trộn Tốt Dộng---Chúc Dộng (cuối năm
1426)},reference={trộn-tốt-dộngchúc-dộng-cuối-năm-1426}]

Tháng 10--1426, 5 vạn viện binh giặc do Vương Thông chỉ huy kéo vào Đông
Quan, nâng số lượng quân Minh ở đây lên tới 10 vạn.

Để giành lại thế chủ động, Vương Thông quyết định mở cuộc phản công lớn,
đánh vào chủ lực của nghĩa quân ở Cao Bộ (Chương Mĩ, Hà Nội).

\startplacefigure[title={Hình 42---Lược đồ trận Tốt Động---Chúc Động}]
{\externalfigure[maps/map2.png]}
\stopplacefigure

Sáng 7-11-1426, Vương Thông cho xuất quân tiến về hướng Cao Bộ.

Nắm được ý đồ và hướng tiến quân của Vương Thông, nghĩa quân đã đặt phục
binh ở Tốt Động và Chúc Động. Khi quân Minh lọt vào trận địa, nghĩa quân
nhâì tề xông thắng vào quân giặc, đánh tan tác đội hình của chúng, dồn
quân giặc xuống cánh đồng lầy lội để tiêu diệt. Kết quả, trên 5 vạn quân
giặc tử thương, bắt sống trên 1 vạn ; Vương Thông bị thương tháo chạy về
Đông Quan ; Thượng thư bộ binh Trần Hiệp cùng các tướng giặc Lý Lượng,
Lý Đằng bị giết tại trận.

\startblockquote
\quotation{Ninh Kỉều\footnote{Ninh Kiều, Tốt Động : -thuộc huyện Chương
  Mĩ, Hà Nội.} máu chảy thành sông, tanh trôi vạn dặm,\crlf
Tốt Động\footnote{Ninh Kiều, Tốt Động : -thuộc huyện Chương Mĩ, Hà Nội.}
thây chát đầy nội, nhơ để ngàn năm.}

(Bình Ngô đại cáo)
\stopblockquote

\startitemize[packed]
\item
  Em hãy trình bày diễn biến trận Tốt Động---Chúc Động (qua lược đổ).
\stopitemize

Sau chiến thắng Tốt Động---Chúc Động, nghĩa quân Lam Sơn thừ a thắng,
vây hãm Đông Quan và giải phóng nhiều châu, huyện.

\stopsectionlevel

\startsectionlevel[title={2. Trộn Chi Lãng---Xương Giong (tháng
10-1427)},reference={trộn-chi-lãngxương-giong-tháng-10-1427}]

Đầu tháng 10-1427, hơn 10 vạn viện binh từ Trung Quốc chia làm hai đạo
kéo vào nước ta. Một đạo do Liễu Thăng chỉ huy, từ Quảng Tây tiến vào
theo hướng Lạng Sơn. Đạo thứ hai do Mộc Thạnh chỉ huy, từ Vân Nam tiến
vào theo hướng Hà Giang.

Bộ chỉ huy nghĩa quân quyết định tập trung lực lượng tiêu diệt viện quân
giặc, trước hết là đạo quân của Liễu Thăng, không cho chúng tiến sâu vào
nội địa nước ta.

Ngày 8-10, Liễu Thăng hùng hổ dẫ n quân ào ạt tiến vào biên giới nước
ta, bị nghĩa quân phục kích và giết ở ải Chi Lăng.

\startblockquote
Khỉ quân Liễu Thăng tiến đêh ải Chi Lăng (Lạng Sơn), nghĩa quân Lam Sơn
do tướng Trần Lựu chỉ huy được lệnh vừ a đánh vừ a lui, nhử địch vào
trận địa phục kích ở ải Chi Lăng. Liễu Thăng thúc quân đuổi theo, lọt
vào trận địa mai phục, lập tức bị quân ta phóng lao đâm chết, quân Minh
hoảng hốt, rối loạn. Nghĩa quân mai phục, do tướng Lê Sát, Lưu Nhân Chú
chỉ huy, thừ a cơ đổ ra đánh, tiêu diệt trên 1 vạn tên giặc.
\stopblockquote

Sau khi Liễu Thăng bị giết, Phó tổng binh là Lương Minh lên thay, chấn
chỉnh đội ngũ, tiến xuống Xương Giang (Bắc Giang). Trên đường tiến quân,
quân giặc liên tiếp bị phục kích ở cần Trạm, Phô” Cát, bị tiêu diệt đến
3 vạn tên, Tổng binh Lương Minh bị giết tại trận, Thượng thư bộ Binh Lý
Khánh phải thắt cổ tự tử.

Mâ'y vạn địch còn lại cố gắng lắm mới tới Xương Giang co cụm lại giữa
cánh đồng, bị nghĩa quân từ nhiều hướng tân công, gần 5 vạn tên bị tiêu
diệt, số còn lại bị bắt sống, kê cả tướng giặc là Thôi Tụ, Hoàng Phúc.

\startblockquote
\quotation{Ngày mười tám\footnote{Ngày 18 tháng 9 năm Đinh Mùi tức ngày
  8-10-1427.}, trận Chi Lăng, Liễu Thăng thất thê', Ngày hai mươi, trận
Mã Yên, Liễu Thăng cụt đầu. Ngày hăm lăm, Bá tước Lương Minh bại trận tử
vong, Ngày hăm tám, Thượng thư Lý Khánh cùng kế tự vẫ n. \ldots{}Đánh
một trận, sạch không kình ngạc, Đánh hai trận, tan tác chim muông
\ldots{}Đô đốc Thôi Tụ lê gối dâng tờ tạ tội, Thượng thư Hoàng Phúc trói
tay để tự xin hàng. Lạng Giang, Lạng Sơn, thây chát đầy dường, Nương
Giang, Bình Than, máu trôi đỏ nước\ldots{}} (Bình Ngô đại cáo)
\stopblockquote

\startplacefigure[title={Hình 43---Lược đồ trận Chi Lăng---Xương Giang}]
{\externalfigure[maps/map3.png]}
\stopplacefigure

Cùng lúc đó, Lê Lợi sai tướng đem các chiến lợi phẩm ở Chi Lăng đến
doanh trại Mộc Thạnh. Mộc Thạnh trông thây, biết Liễu Thăng đã bại trận
nên vô cùng hoảng sợ, vội vàng rút chạy về Trung Quốc.

\startitemize[packed]
\item
  Dựa vào lược đô', em hãy trình bày diễn biến trận Chi Lăng---Xương
  Giang.
\stopitemize

Được tin hai đạo viện binh Liễu Thăng, Mộc Thạnh đã bị tiêu diệt, Vương
Thông ở Đông Quan vô cùng khiếp đảm, vội vàng xin hoà và chấp nhận mỏ
hội thề Đông Quan (ngày 10-12-1427) để được an toàn rút quân về nước.
Ngày 3-1-1428, toán quân cuối cùng củạ Vương Thông rút khỏi nước ta. Đất
nước sạch bóng quân thù.

\stopsectionlevel

\startsectionlevel[title={3. Nguyên nhân thắng lợi và ý nghĩa lịch
sử},reference={nguyên-nhân-thắng-lợi-và-ý-nghĩa-lịch-sử}]

Đất nước được giải phóng hoàn toàn, Nguyễn Trãi viết bài Bình Ngô đại
cáo. Đây là một áng anh hùng ca tổng kết hết sức tài tình cuộc kháng
chiến vĩ đại của dân tộc từ những ngày gian lao ở núi Chí Linh đến các
chiến thắng lẫ y lừ ng Tốt Động---Chúc Động, Chi Lăng---Xương Giang.

\startblockquote
Bình Ngô đại cáo không những nêu bật ý nghĩa lịch sử to lớn của cuộc
khỏi nghĩa Lam Sơn : \quotation{Xã tắc từ đây vững bền, Giang sơn từ đây
đổi mới}, mà còn toát lên niềm tự hào dân tộc sâu sắc, chủ nghĩa yêu
nước và nhân đạo sáng ngời \quotation{Đem đại nghĩa để thắng hung tàn,
Lấy chí nhân đê thay cường bạo} của nhân dân ta trong cuộc khởi nghĩa
đó.
\stopblockquote

Cuộc khởi nghĩa Lam Sơn thắng lợi vẻ vang là do nhân dân ta có lòng yêu
nước nồng nàn, ý chí bất khuất quyết tâm giành lại độc lập tự do cho đất
nước, toàn dân đoàn kết chiến đấu. Tất cả các tầng lớp nhân dân không
phân biệt nam nữ, già trẻ, các thành phần dân tộc đều đoàn kết đánh
giặc, hăng hái tham gia kháng chiến (gia nhập lực lượng vũ trang, tự vũ
trang đánh giặc, ủng hộ, tiếp tế lương thực cho nghĩa quân v.v\ldots{}).

Thắng lợi của cuộc khởi nghĩa Lam Sơn gắn liền với đường lối chiến lược,
chiến thuật đúng đắn, sáng tạo của bộ tham mưu, đứng đầu là các anh hùng
dân tộc Lê Lợi, Nguyễn Trãi. Những người lãnh đạo cuộc khởi nghĩa đã
biết dựa vào dân, từ cuộc khởi nghĩa phát triển thành cuộc chiến tranh
giải phóng dân tộc quy mô cả nước, hoàn thành thắng lợi nhiệm vụ giải
phóng đất nước.

Cuộc khởi nghĩa Lam Sơn thắng lợi đã kết thúc 20 năm đô hộ tàn bạo của
phong kiến nhà Minh, mở ra một thời kì phát triển mới của xã hội, đất
nước, dân tộc Việt Nam - thời Lê sơ.

\stopsectionlevel

\startsectionlevel[title={CÂU HỎI},reference={câu-hỏi-2}]

\startitemize[n,packed][stopper=.]
\item
  Dựa vào các lược đồ và bài học, em hãy trình bày tóm tắt diễn biến
  cuộc khởi nghĩa Lam Sơn.
\item
  Hãy nêu những nguyên nhân thắng lợi của cuộc khởi nghĩa Lam Sơn.
\item
  Cuộc khởi nghĩa Lam Sơn thắng lợi có ý nghĩa lịch sử gì?
\stopitemize

\stopsectionlevel

\stopsectionlevel

\stoptext
